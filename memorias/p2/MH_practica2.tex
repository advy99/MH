\documentclass[12pt, spanish]{article}
\usepackage[spanish]{babel}
\selectlanguage{spanish}
\usepackage{natbib}
\usepackage{url}
\usepackage[utf8x]{inputenc}
\usepackage{graphicx}
\graphicspath{{images/}}
\usepackage{parskip}
\usepackage{fancyhdr}
\usepackage{vmargin}
\usepackage{multirow}
\usepackage{float}
\usepackage{chngpage}

\usepackage{subcaption}

\usepackage{hyperref}
\usepackage[
    type={CC},
    modifier={by-nc-sa},
    version={4.0},
]{doclicense}

\hypersetup{
    colorlinks=true,
    linkcolor=blue,
    filecolor=magenta,      
    urlcolor=cyan,
}

% para codigo
\usepackage{listings}
\usepackage{xcolor}



%% configuración de listings

\definecolor{listing-background}{HTML}{F7F7F7}
\definecolor{listing-rule}{HTML}{B3B2B3}
\definecolor{listing-numbers}{HTML}{B3B2B3}
\definecolor{listing-text-color}{HTML}{000000}
\definecolor{listing-keyword}{HTML}{435489}
\definecolor{listing-identifier}{HTML}{435489}
\definecolor{listing-string}{HTML}{00999A}
\definecolor{listing-comment}{HTML}{8E8E8E}
\definecolor{listing-javadoc-comment}{HTML}{006CA9}

\lstset{language=C++}

\lstdefinestyle{eisvogel_listing_style}{
%  language = C,
%$if(listings-disable-line-numbers)$
%  xleftmargin      = 0.6em,
%  framexleftmargin = 0.4em,
%$else$
  numbers          = left,
  xleftmargin      = 0em,
 framexleftmargin = 0em,
%$endif$
  backgroundcolor  = \color{listing-background},
  basicstyle       = \color{listing-text-color}\small\ttfamily{}\linespread{1.15}, % print whole listing small
  breaklines       = true,
  frame            = single,
  framesep         = 0.19em,
  rulecolor        = \color{listing-rule},
  frameround       = ffff,
  tabsize          = 4,
  numberstyle      = \color{listing-numbers},
  aboveskip        = 1.0em,
  belowskip        = 0.1em,
  abovecaptionskip = 0em,
  belowcaptionskip = 1.0em,
  keywordstyle     = \color{listing-keyword}\bfseries,
  classoffset      = 0,
  sensitive        = true,
  identifierstyle  = \color{listing-identifier},
  commentstyle     = \color{listing-comment},
  morecomment      = [s][\color{listing-javadoc-comment}]{/**}{*/},
  stringstyle      = \color{listing-string},
  showstringspaces = false,
  escapeinside     = {/*@}{@*/}, % Allow LaTeX inside these special comments
  literate         =
  {á}{{\'a}}1 {é}{{\'e}}1 {í}{{\'i}}1 {ó}{{\'o}}1 {ú}{{\'u}}1
  {Á}{{\'A}}1 {É}{{\'E}}1 {Í}{{\'I}}1 {Ó}{{\'O}}1 {Ú}{{\'U}}1
  {à}{{\`a}}1 {è}{{\'e}}1 {ì}{{\`i}}1 {ò}{{\`o}}1 {ù}{{\`u}}1
  {À}{{\`A}}1 {È}{{\'E}}1 {Ì}{{\`I}}1 {Ò}{{\`O}}1 {Ù}{{\`U}}1
  {ä}{{\"a}}1 {ë}{{\"e}}1 {ï}{{\"i}}1 {ö}{{\"o}}1 {ü}{{\"u}}1
  {Ä}{{\"A}}1 {Ë}{{\"E}}1 {Ï}{{\"I}}1 {Ö}{{\"O}}1 {Ü}{{\"U}}1
  {â}{{\^a}}1 {ê}{{\^e}}1 {î}{{\^i}}1 {ô}{{\^o}}1 {û}{{\^u}}1
  {Â}{{\^A}}1 {Ê}{{\^E}}1 {Î}{{\^I}}1 {Ô}{{\^O}}1 {Û}{{\^U}}1
  {œ}{{\oe}}1 {Œ}{{\OE}}1 {æ}{{\ae}}1 {Æ}{{\AE}}1 {ß}{{\ss}}1
  {ç}{{\c c}}1 {Ç}{{\c C}}1 {ø}{{\o}}1 {å}{{\r a}}1 {Å}{{\r A}}1
  {€}{{\EUR}}1 {£}{{\pounds}}1 {«}{{\guillemotleft}}1
  {»}{{\guillemotright}}1 {ñ}{{\~n}}1 {Ñ}{{\~N}}1 {¿}{{?`}}1
  {…}{{\ldots}}1 {≥}{{>=}}1 {≤}{{<=}}1 {„}{{\glqq}}1 {“}{{\grqq}}1
  {”}{{''}}1
}
\lstset{style=eisvogel_listing_style}


\usepackage[default]{sourcesanspro}

\setmarginsrb{2 cm}{1 cm}{2 cm}{2 cm}{1 cm}{1.5 cm}{1 cm}{1.5 cm}

\title{Práctica 2:\\
Algoritmos basados en poblaciones  \hspace{0.05cm} }                           
\author{Antonio David Villegas Yeguas}                             
\date{\today}                                           

\renewcommand*\contentsname{hola}

\makeatletter
\let\thetitle\@title
\let\theauthor\@author
\let\thedate\@date
\makeatother

\pagestyle{fancy}
\fancyhf{}
\rhead{\theauthor}
\lhead{\thetitle}
\cfoot{\thepage}

\begin{document}

%%%%%%%%%%%%%%%%%%%%%%%%%%%%%%%%%%%%%%%%%%%%%%%%%%%%%%%%%%%%%%%%%%%%%%%%%%%%%%%%%%%%%%%%%

\begin{titlepage}
    \centering
    \vspace*{0.3 cm}
    \includegraphics[scale = 0.50]{ugr.png}\\[0.7 cm]
    %\textsc{\LARGE Universidad de Granada}\\[2.0 cm]   
    \textsc{\large 3º CSI 2019/20 - Grupo 1}\\[0.5 cm]            
    \textsc{\large Grado en Ingeniería Informática}\\[0.5 cm]              
    \rule{\linewidth}{0.2 mm} \\[0.2 cm]
    { \huge \bfseries \thetitle}\\
    \rule{\linewidth}{0.2 mm} \\[1 cm]
    
    \begin{minipage}{0.4\textwidth}
        \begin{flushleft} \large
            \emph{Autor:}\\
            \theauthor\\ 
			 \emph{DNI:}\\
            77021623-M
            \end{flushleft}
            \end{minipage}~
            \begin{minipage}{0.4\textwidth}
            \begin{flushright} \large
            \emph{Asignatura: \\
            Metaheurísticas}   \\     
            \emph{Correo:}\\
            advy99@correo.ugr.es           
        \end{flushright}
    \end{minipage}\\[0.5cm]
  
    {\large \thedate}\\[0.5cm]
    {\url{https://github.com/advy99/MH/}}
    {\doclicenseThis}
 	
    \vfill
    
\end{titlepage}

%%%%%%%%%%%%%%%%%%%%%%%%%%%%%%%%%%%%%%%%%%%%%%%%%%%%%%%%%%%%%%%%%%%%%%%%%%%%%%%%%%%%%%%%%

\tableofcontents
\pagebreak

%%%%%%%%%%%%%%%%%%%%%%%%%%%%%%%%%%%%%%%%%%%%%%%%%%%%%%%%%%%%%%%%%%%%%%%%%%%%%%%%%%%%%%%%%


\section{Descripción del problema de la asignación con restricciones.}

El problema de la asignación con restricciones consiste en una generalización del problema de agrupamiento clásico, bastante común en \textit{Machine Learning}.

El problema del agrupamiento clásico es un problema en el que se recibe como entrada las características de un conjunto de elementos y el número de agrupaciones a realizar, para resolver el problema tendremos como objetivo realizar dichas agrupaciones de los distintos elementos con el fin de organizarlos acorde a las características dadas. Llamaremos a estas agrupaciones \textit{Clusters}.

Como extensión a este problema, nosotros trabajaremos sobre el problema de asignación con restricciones (de ahora en adelante PAR). PAR se basa en el problema del agrupamiento clásico, pero añadiendo al problema restricciones entre los propios elementos, es decir, las distintas parejas que podemos formar con los datos tendrán asociadas restricciones de dos tipos:

\begin{itemize}
	\item{Must-Link (ML): Dados dos datos $D_1$ y $D_2$, si estos datos tienen asociada una restricción ML deberán tener asignado el mismo cluster.}
	\item{Cannot-Link (CL): Dados dos datos $D_1$ y $D_2$, si estos datos tienen asociada una restricción CL deberán tener asignado distinto cluster.}
\end{itemize}

Nosotros trataremos estas restricciones como restricciones débiles, es decir, permitiremos que se incumplan pero penalizándolas, por lo tanto, el objetivo para resolver el PAR es minimizar la distancia de los elementos que conforman los distintos clusters, así como asegurarnos que no se incumple ninguna restricción.

Otra restricción fuerte del problema es que todos los clusters tienen que tener al menos un elemento.

Para comprobar que la distancia entre los elementos de los distintos clusters es mínima, tendremos las siguientes características asociadas a un cluster:

\begin{itemize}
	\item{Centroide: Valor promedio de los datos que conforman el cluster. Con este elementos obtendremos la representación del elemento central del cluster.}
	\item{Distancia media intra-cluster: Con este elemento mediremos como de disperso está nuestro cluster, es decir, si los elementos de un cluster están cercanos entre si.}
\end{itemize}

Además también contaremos con distintas características del PAR:

\begin{itemize}
	\item{Desviación general: Será la media de las desviaciones intra-cluster de los distintos clusters que conforman el PAR. Uno de nuestros objetivos será que este valor sea mínimo.}
	\item{\textit{Infeasibility}: Esta característica nos permitirá conocer cuantas restricciones se están incumpliendo en una posible solución del PAR. Otro de nuestros objetivos será que este valor sea mínimo.}
\end{itemize}

\newpage

\section{Descripción de la implementación común y representación del problema para su resolución.}

Para el desarrollo e implementación del programa que resolverá el PAR he decidido usar C++ como lenguaje de programación.

Para la representación del problema he decidido construir dos clases en C++, la clase PAR y la clase Cluster.



\subsection{Clase PAR:}


Esta clase contará con los siguientes atributos:

\begin{itemize}
	\item {Matriz que almacenará valores reales, donde estarán las características de los datos.}
	\item {Vector de objetos tipo Cluster con el que representaremos las agrupaciones a conseguir.}
	\item {Diccionario con las restricciones entre los distintos datos.}
	\item {Desviación general del problema.}
	\item {Mayor distancia entre dos los distintos datos del problema.}
	\item {\textit{Infeasibility} del problema}
\end{itemize}


\subsection{Clase Cluster:}

Con esta clase representaremos los elementos y operaciones internas de un cluster. Tendrá los siguientes atributos:

\begin{itemize}
	\item {Set que almacenará enteros, representando que elementos conforman dicho cluster.}
	\item {Vector de reales que representará el centroide.}
	\item {Valor real que representará la distancia intra-cluster de los elementos que lo conforman.}
	\item {Una referencia a la clase PAR asociada, con la que obtendremos los datos necesarios sin necesidad de duplicarlos.}
\end{itemize}

Es importante mencionar que la declaración de la clase Cluster se realiza dentro de la clase PAR, ya que no tiene sentido crear un cluster sin tener un problema asociado.


\newpage

\subsection{Representación:}

\subsubsection{Datos:}

Con respecto a la representación del problema, los datos son almacenados en una matriz de datos tipo \texttt{double}, cada fila representará un dato, y las columnas representarán las características de dicho dato.

\subsubsection{Restricciones:}

Las restricciones serán almacenadas en un diccionario, donde la clave será la pareja de datos afectada por la restricción y el valor será 1 si la restricción es ML o -1 si es CL. He decidido usar esta representación ya que nos permite almacenar la información de forma eficiente sin tener que almacenar los elementos que no tiene restricciones entre sí, nos permite acceder a los elementos de una forma más rápida que con una lista (aunque no tanto comparado con una matriz, pero gracias al operador \texttt{find} de la clase esto apenas se nota) y podemos recorrer las restricciones secuencialmente de una forma rápida gracias a los iteradores disponibles en la clase diccionario de C++.


Con esta representación de los datos y las restricciones, las distintas operaciones antes comentadas se pueden resolver de forma sencilla generalizando el número de generalizando el número de características, independientemente del problema.


\subsubsection{Solución:}

Para representar la solución he decido, como he comentado anteriormente, que la solución esté compuesta por un vector de objetos tipo cluster, a pesar de que esta representación no será valida para futuras prácticas, es mucho más sencillo y práctico trabajar con ella en la primera práctica, ya que al separar los distintos clusters en su propio objeto podemos realizar las operaciones que afecten a los clusters de forma mucho más rápida y permitiendo la factorización del problema, ya que por ejemplo, seremos capaces de recalcular el centroide de un cluster sin tener que tener en cuenta los demás, o separar los elementos.

La representación dada en clase (vector de enteros de tamaño N siendo N el número de datos, y cada posición del vector se le asocia un número, que será el cluster al que pertenece) es de gran utilidad en la práctica 2, por lo que tenemos una función que nos intercambiará entre estas soluciones, es decir, una función que dado un vector de clusters nos devolverá un vector de enteros con las asignaciones del argumento dado.

\begin{lstlisting}
//vector de ints en el que devolveremos la solución
vector_sol 

Para todo cluster i en el vector de clusters:
	Para todo elemento j en el cluster i:
		vector_sol[j] = i;
		
Devolver vector_sol

\end{lstlisting}


\subsubsection{Operaciones sobre los clusters:}

Los clusters tendrán principalmente dos funciones que podremos realizar:

\begin{itemize}
	\item {Calcular el centroide}
	\item {Calcular la distancia media intra-cluster}
\end{itemize}

Para calcular el centroide basta con recorrer los elementos que conforman ese cluster (disponibles dentro de la clase Cluster) y calcular el punto medio.

\begin{lstlisting}
Para i = 0 hasta el tamaño del centroide (numero de caracteristicas de un dato) :
	centroide[i]=0.0d
	
Para cada elemento i del cluster:
	Para cada carasterictica j del elemento i:
		centroide[j] += problema.datos[i][j];
		
Para cada caracteristica i del centroide:
	centroide[i] /= num_elementos_cluster 
	
	
\end{lstlisting}

Para calcular la distancia media intra-cluster calcularemos el centroide y tras eso la distancia de todos los elementos al centroide.


\begin{lstlisting}
calcular_centroide();
distancia_intra_cluster = 0
	
Para cada elemento i del cluster:
	Para cada carasterictica j del elemento i:
		distancia_intra_cluster += problema.datos[i][j];
		
Para cada caracteristica i del centroide:
	distancia_intra_cluster /= num_elementos_cluster 
	
	
\end{lstlisting}

\newpage

\subsubsection{Operaciones sobre PAR:}

Sobre el problema podremos aplicar las siguientes operaciones:

\begin{itemize}
	\item {Calcular la desviación general.}
	\item {Buscar el cluster en el que se encuentra cierto elemento.}
	\item {Calcular \textit{Infeasibility} del estado actual del PAR.}
	\item {Calcular la distancia entre dos puntos del problema.}
\end{itemize}

Para calcular la desviación general haremos la media de las distintas desviaciones intra-cluster.


\begin{lstlisting}
desviacion_general = 0
	
Para cada elemento i del vector de clusters:
	clusters[i].calcular_desviacion_intra_cluster()
	desviacion_general += cluster[i].desviacion_intra_cluster()
		
desviacion_general /= clusters.size()	
	
\end{lstlisting}

Para buscar un elemento N dado por parámetro en los distintos clusters simplemente recorreremos el vector de clusters usando el operador find de la clase set, al estar ordenados y ser únicos, esto hará que esta operación sea muy rápida.

\begin{lstlisting}
encontrado = false
encontrado_en_cluster = -1

Mientras !encontrado Y para cada elemento i del vector clusters
	Si clusters[i].elementos.find(N) != clusters[i].elementos.end()
		encontrado = true	
		encontrado_en_cluster = i
		
return encontrado_en_cluster
	
\end{lstlisting}


Para calcular  \textit{Infeasibility} del PAR simplemente recorreremos el diccionario de restricciones, si encontramos dos elementos en distinto cluster y son ML sumamos 1 al total, y si encontramos dos elementos en el mismo cluster y tienen la restricción de CL sumamos 1 al total.

Solo comprobaremos si en el diccionario, la pareja de datos el primer elemento es mayor que el segundo. Esto lo hacemos para evitar contar dos veces la misma restricción, por ejemplo, si tenemos la restricción 0 con 1: ML, también tenemos la 1 con 0: ML, así que solo comprobaremos la 1 con 0. A pesar de usar el diccionario he decidido duplicar de esta forma las restricciones ya que a veces necesitaremos acceder sin tener en cuenta el orden de los datos.

\newpage

\begin{lstlisting}
infac = 0

Para cada elemento i de  restricciones
	Si i.first.first > i.first.second
		Si existe una restricción entre i.first.first e i.first.second
			c1 = buscar_elemento(i.first.first)
			c1 = buscar_elemento(i.first.second)
			
			Si c1 == c2 Y i.second == -1
				infac++
			Si c1 != c2 Y i.second == 1
				infac++
		
		
return infac
	
	
\end{lstlisting}




Tanto en PAR como en Cluster tendremos otras operaciones auxiliares relativas a los algoritmos de búsqueda o comparación, que explicaré más adelante.



\subsubsection{Cambio de representación.}

Como hemos comentado, la representación usada es un vector de objetos tipo Cluster, y bajo esa representación hemos resuelto la práctica 1 en la que usamos un algoritmo greedy y una búsqueda local, sin embargo, para trabajar con los algoritmos basados en poblaciones propuestos para la práctica 2 (Algoritmos Genéticos y Meméticos) esta representación no es válida, ya que no disponemos de una forma de cruzar las soluciones de datos, por lo tanto, planteo una segunda representación, en la que una solución será un vector de N posiciones, siendo N el número de elementos del problema, donde cada posición almacena un entero en el intervalo $[0, num_clusters]$ que indicará que cluster tiene asignado el elemento N.


\begin{figure}[H]
  \centering
      $$[0, 1, 2, ... , 2, 1, 1]$$
 		 \caption{Representación de una solución, donde al primer elemento asignamos el cluster 0 y al último el cluster 1}
  		\label{fig:ej_nueva_sol}
\end{figure}

Vemos como esto nos puede traer problemas a la hora de calcular la función objetivo, porque es mucho más costoso actualizar los centroides, calcular la distancia media intra-cluster, etc, sin embargo esto lo solucionaremos de una forma muy simple, tendremos dos funciones, para pasar de una representación a otra, de forma que realizaremos las distintas operaciones en la representación más cómoda para cada operación.

\newpage

\textbf{Paso de vector de clusters a nueva representación:}

\begin{lstlisting}
Devuelve un vector de enteros - recibe un vector de clusters "clusters":
	
	tam_solucion = 0
	
	Para cada cluster i en clusters:
		tam_solucion += i.size()
		
	solucion = vector de enteros (tamaño = tam_solucion)
	
	Para cada cluster i en clusters:
		Para cada elemento j dentro del cluster i:
			solucion[j] = i
			
	devolver solucion
\end{lstlisting}

De esta forma, a cada elemento le asociamos en su correspondiente posición el valor del cluster al que pertenece.

\textbf{Paso de nueva representación a vector de clusters:}

De forma similar podemos hacer la operación contraria:

\begin{lstlisting}
Devuelve un vector de clusters - recibe un vector de enteros solucion_enteros
	solucion = vector de clusters (el númer de clusters lo tenemos almacena en el problema)
	
	Con i desde 0 hasta solucion_enteros.size():
		solucion[solucion_enteros[i]].add_elemento(i)
	
	devolver solucion
\end{lstlisting}

De esta forma, que \texttt{solucion\_enteros[i]} nos dice el cluster donde se encuentra \texttt{i}, luego simplemente lo añadimos a ese cluster en la representación de vector de clusters.



Como hemos comentado, cuando queramos aplicar una operación que sea costosa sobre una representación, pero simple sobre otra, simplemente cambiaremos de una a otra usando estas funciones.

\newpage

\subsection{Calcular \textit{Infeasibility} parcial.}

Para hacer una factorización, además de poder calcular los atributos de los clusters de forma independiente sin necesidad de recalcular la de todos si no son modificados también necesitaremos una función que dado un cluster y un elemento nos calcula cuantas restricciones incumpliría si lo introducimos en dicho cluster.


\begin{lstlisting}
elemento: nos lo dan como argumento
cluster: nos lo dan como argumento

incumplidas = 0


Para todos los elementos i del cluster dado:
	Si i y elemento tienen una restricción Y dicha restricción == -1
		incumplidas++

Para los elementos j de los clusters != cluster:
	Si j y elemento tienen una restricción Y dicha restricción == 1
		incumplidas++



return incumplidas
	
	
\end{lstlisting}


Esta factorización nos servirá para el algoritmo greedy y el algoritmo de búsqueda local, sin embargo no nos servirá con los nuevos algoritmos de la práctica 2, ya que al modificar una solución, esta será totalmente distinta por lo que tendremos que reevaluarla de nuevo.

\newpage

\subsection{Función objetivo.}

En el caso del PAR nuestro objetivo será agrupar los datos en clusters incumpliendo el mínimo de restricciones.

Para cumplir la primera parte intentaremos que los datos estén lo menos dispersos con respecto a su centroide, por lo que intentaremos minimizar el total de las desviaciones intra-cluster, en resumen, \textbf{minimizar la desviación general}. 

Para la segunda parte, intentaremos minimizar el número de restricciones incumplidas penalizando las soluciones que más restricciones incumplan. Esta penalización se basará en sumar a la desviación general un valor entre 0 y la distancia más grande entre los datos del problema.

Esto lo conseguiremos con este factor, al que llamaremos $\lambda$:

$$ \lambda = \frac{D_{max}}{NumRestricciones} $$ 


De forma que si incumplimos todas las restricciones la solución la consideraremos mucho peor que otra con mayor desviación general pero menor restricciones incumplidas.

Nuestra función objetivo será:

$$ f = C + (\textit{Infeasibility} * \lambda) $$ 

Dicho esto podemos implementar la siguiente función:

\begin{lstlisting}
Devolvemos un double:
	devolver desviacion_general + (calcular_infactibilidad() * get_lambda())
\end{lstlisting}

Teniendo ya Lambda calculado al ser un valor que conocemos desde el inicio del problema.

\newpage

\section{Métodos de resolución del problema.}

A continuación explicaré los distintos métodos de resolución del problema vistos en teoría y prácticas. Los siguientes pseudocógidos no tienen en cuenta variables de salida para recabar datos de trazas o salidas de la ejecución.

\subsection{Algoritmo de Búsqueda Local.}

Este algoritmo se basa en partir de una solución inicial aleatoria, a partir de esa solución explorar una solución vecina, y si esta solución vecina es mejor que la actual, moverse a dicha solución.

Este algoritmo no nos asegurará obtener el optimo, solo nos garantiza obtener un mínimo local (puede que este mínimo local sea el optimo, pero no lo podemos asegurar).

El algoritmo solo cambiará una solución por otra mejor, por lo que el punto de inicio será muy importante.

Para desarrollar este algoritmo necesitamos varios componentes:
 
 \subsubsection{Generador de solución inicial aleatoria.}
 
 Primero he desarrollado una función que nos genera una solución aleatoria que cumple con las restricciones fuertes (todos los clusters tienen al menos un elemento).
 
 \begin{lstlisting}
vaciar_clusters()

indices_aleatorios = {0...datos.size()}

indices_aleatorios = random_shuffle(indices)

contador = 0

// nos aseguramos que cada cluster tiene uno al menos
Para cada indice i desde 0 hasta clusters.size():
	clusters[i].insertar_elemento(indices_aleatorios[contador])
	contador++
	
Para contador < indices_aleatorios.size()
	clusters[EnteroAleatorio(0, clusters.size())].insertar_elemento(indices_aleatorios[contador])
	contador++
	

\end{lstlisting}
 
 
\subsubsection{Generador de vecinos}

En nuestro caso, construiremos el vecindario a partir de una solución, modificando un único elemento de cluster, siempre que no se queden clusters vacíos, es decir, si tenemos una solución S, generaremos un vecino $S_1$ eliminando un elemento de un cluster $i$ e insertandolo en otro cluster $j$, siempre que el cluster $i$ no se quede vacío y $i$ sea distinto a $j$.

El elemento a mover y el nuevo cluster lo escogeremos de forma aleatoria, como explicaré más adelante en el desarrollo del algoritmo completo.

 
\subsubsection{Función objetivo} 
 
Una vez tenemos una primera solución, necesitamos conocer como obtener el valor de la función objetivo asociada a esa función.

Cada vez que generemos un vecino recalcularemos su desviación general, por lo que calcular su función objetivo será:

\begin{lstlisting}
funcion_objetivo = get_desviacion_promedio + ( LAMBDA * calcular_infactibilidad() )
 \end{lstlisting}


Sin embargo, para reducir el tiempo de ejecución vamos a factorizar el cálculo de \textit{Infeasibility}, ya que dado un \textit{Infeasibility} de X, el valor para cualquier vecino será:

$$X - \textit{Infeasibility}\_cambios\_salida + \textit{Infeasibility}\_cambios\_entrada$$
 
 Al solo modificar un elemento por vecino, \textit{Infeasibility}\_cambios\_salida  es el número de restricciones que incumplía dicho elemento en el cluster antiguo y \textit{Infeasibility}\_cambios\_entrada es el número de restricciones que incumple el nuevo cluster.
 
 
 
\begin{lstlisting}
infact -= cumple_restricciones(elemento, antiguo_num_cluster)
infact += cumple_restricciones(elemento, nuevo_num_cluster)

funcion_objetivo = get_desviacion_promedio + ( LAMBDA * calcular_infactibilidad() )
 \end{lstlisting}

\subsubsection{Criterio de aceptación.}

Se aceptará un vecino si su función objetivo es menor que la de la solución actual, es decir, seguiremos una estrategia primero el mejor, en cuanto tengamos un mejor candidato nos movemos a el, en lugar de generar todo el vecindario y quedarnos con el mejor del vecindario.

Finalizaremos la búsqueda si no encontramos mejor vecino en todo el vecindario.

\subsubsection{Exploración del entorno.}

La exploración del entorno se realizará de forma aleatoria, cada iteración (no evaluación) reordenaremos de forma aleatoria los indices con los que recorreremos los elementos y los indices con los que recorreremos los clusters. Cada elemento lo probaremos con los distintos clusters (a excepción del propio), por lo que iremos generando los vecinos de cada elemento sin llegar a generar el vecindario completo, solo generamos vecindario hasta que encontramos un mejor vecino. Lo veremos más adelante en el algoritmo.

\subsubsection{Desarrollo del algoritmo.}

El algoritmo de búsqueda local se basará en generar una solución aleatoria, a partir de esa solución aleatoria generar vecinos, evaluarlos hasta que encuentre un mejor vecino, en cuanto encuentre un mejor vecino moverse a este, y parar en caso de que en todo el vecindario no encuentre un mejor vecino o se llegue al límite de evaluaciones, en nuestro caso 100.000.


Con las distintas operaciones podemos codificar el algoritmo de la siguiente forma:

{\small
\begin{lstlisting}
generar_solucion_aleatoria()
calcular_desviacion_general()

LAMBDA = mayor_distancia / restricciones.size()
evaluaciones = 0
encontrado_mejor = false
indices = {0..datos.size()}
indices_clusters = {0..clusters.size()}
infac = calcular_infactibilidad()
infac_vecino = infac
f_objetivo = get_desviacion_general() + (infac * LAMBDA)
f_objetivo_vecino = 0

do:
	encontrado_mejor = false
	reordenar_aleatoriamente(indices)
	reordenar_aleatoriamente(indices_clusters)
	Para cada elemento i de indices Y !encontrado_mejor
		Para cada elemento j del indices_clusters Y !encontrado_mejor
			antiguo_cluster = buscar_elemento(i)
			Si antiguo_cluster != j Y clusters[antiguo_cluster].size() - 1 > 0
				clusters[antiguo_cluster].eliminar_elemento(i)
				infac_vecino -= incumple_restricciones(i, antiguo_cluster)
				infac_vecino += incumple_restricciones(i, j)
				clusters[j].añadir_elemento(i)
				calcular_desviacion_general()
				f_objetivo_vecino = get_desviacion_general() + (infac_vecino * LAMBDA)
				evaluaciones++				
				
				Si f_objetivo_vecino < f_objetivo
					f_objetivo = f_objetivo_vecino
					infac = infac_vecino
					encontrado_mejor = true
				Si NO
					clusters[j].eliminar_elemento(i)
					clusters[antiguo_cluster].añadir_elemento(i)
					infact_vecino = infac

while evaluaciones < TOPE_BL Y encontrado_mejor
 \end{lstlisting}
 }
 
 
Como consideraciones a este algoritmo, vemos como por el operador de vecino es posible que nos estanquemos en mínimos locales de los que no seamos capaces de salir, siendo este el principal problema del algoritmo de búsqueda local.

También mencionar que el escoger la función objetivo puede hacer variar la solución de una forma bastante importante dependiendo del peso que le demos a incumplir las restricciones, como veremos más adelante en los análisis de los resultados y experimentos, ya que de la función objetivo dependerá la penalización por incumplir restricciones por lo que modificando esta función podremos escoger entre mejorar la desviación general o no permitir que se incumplan restricciones.

\newpage

\subsection{Algoritmos evolutivos basados en poblaciones.}

\subsubsection{Introducción a los algoritmos evolutivos.}

Los algoritmos evolutivos basados en poblaciones, como hemos visto en teoría, son algoritmos que se basan en tener un trabajar con un conjunto de soluciones al que llamaremos población, con el objetivo de aplicar distintos operadores para mejorar estas soluciones de la población. A cada una de las soluciones de la población la llamaremos cromosoma, es decir, una población está formada por cromosomas.

De forma similar, como hemos comentado en la nueva representación, la \textbf{representación usada en la práctica 1} (búsqueda local y el algoritmo greedy) \textbf{no nos servirá aquí}, ya que las operaciones que vamos a aplicar a los cromosomas no se pueden aplicar ya que, como explicaremos más adelante, no sería viable obtener la información ni una forma de aplicar estos cambios. Con la nueva representación, usando un vector de enteros en el que el elemento N se representa con la posición N del vector, podemos hablar de genes, que serán los elementos de la solución, los elementos que conforman a un cromosoma.

Como resumen, en un algoritmo evolutivo encontraremos los siguientes elementos:

\begin{enumerate}
	\item \textbf{Gen}: Representación de un elemento de una solución. En la representación un elemento tendrá asociado el cluster al que pertenece.
	\item \textbf{Cromosoma}: Representación de una solución, compuesto por genes.
	\item \textbf{Población}: Representación de un conjunto de cromosomas, es decir, conjunto de soluciones.
\end{enumerate}

El objetivo de esta práctica será ver como con una población, haciendo que los distintos cromosomas interactúen, conseguiremos explorar el espacio de búsqueda, obteniendo soluciones diversas, a la vez que intentaremos que las interacciones entre los cromosomas nos lleven a mejores soluciones intentando seguir el modelo evolutivo, en el que a partir de los cromosomas obtendremos sus hijos, que contendrán información de ambos padres, lo que hará que las nuevas soluciones intenten tomar la mejor parte de ambos a la vez que obtendremos soluciones diversas, evitando los problemas de la búsqueda local, cuyo problema era estancarse en mínimos locales, estos problemas los evitaremos ya que la exploración es mayor como comentaremos más adelante.


\subsubsection{Esquema de los algoritmos evolutivos.}

En esta práctica se nos pide implementar dos tipos de algoritmos:

\begin{itemize}
	\item {Algoritmos Genéticos.}
	\item {Algoritmos Meméticos.}
\end{itemize}

De los cuales vamos a implementar distintas versiones que explicaremos más adelante, sin embargo todos los algoritmos y variaciones seguirán el mismo esquema general de evolución (cada uno tendrá distintas modificaciones, que explicaremos en el desarrollo especifico de cada uno).

\newpage

El esquema evolutivo consta de 4 pasos:

\begin{enumerate}
	\item \textbf{Selección}: A partir de la población actual, seleccionamos a los elementos que formarán parte del proceso evolutivo. Esta selección se suele aplicar usando una ruleta o como en nuestro caso, un torneo binario.
	\item \textbf{Cruce}: Agruparemos los cromosomas de la población por parejas, y aplicando una probabilidad de cruce decidiremos si cruzar las parejas. Existen diversas formas de aplicar el cruce y más adelante explicaremos dos de las que utilizaremos. Un cruce consiste en, dados dos cromosomas (soluciones) crear dos cromosomas hijos, cada uno de estos hijos estará compuesto por una mezcla de sus padres. La idea es dados dos padres que se comportan bien como solución generen hijos que al usar la información del primer padre sumada a la información del segundo, se comporten mejor.
	\item \textbf{Mutación}: Tras aplicar el cruce existirá una pequeña probabilidad de que los genes que conforman los cromosomas muten, es decir, cambien de valor de forma aleatoria. Más adelante explicaremos un método para conseguir esto.
	\item \textbf{Reemplazamiento}: Tras aplicar la selección, el cruce y la mutación, los cromosomas resultantes reemplazarán a la población anterior. Existen varías formas de aplicar el reemplazamiento, ligadas a la forma de realizar la selección y si se decide hacer con elitismo, es decir, mantener el mejor cromosoma de la población en la nueva población.
\end{enumerate}

Para realizar este esquema de evolución necesitamos distintos operadores, que pasaré a explicar en las siguientes secciones.

\subsubsection{Algoritmos evolutivos a implementar.}

En esta práctica vamos a implementar tres algoritmos (sin contar las variaciones de cruce):

\begin{enumerate}
	\item Algoritmos Genéticos Generacionales: Estos algoritmos se basarán totalmente en el esquema de evolución antes comentado, y cada población seleccionada, cruzada y mutada sustituirá por completo a la generación actual. Ejecutaremos estos algoritmos con un tamaño de población de 50 cromosomas, una probabilidad de cruce de 0'7 y una probabilidad de mutación del 0'001. Existirán dos variaciones, usando dos operadores de cruce distintos.
	\item Algoritmos Genéticos Estacionarios: Estos algoritmos también se basarán totalmente en el esquema de evolución, sin embargo, la población seleccionada será de únicamente dos elementos, por lo que en cada generación, gran parte de la población se mantendrá y solo cambiarán dos cromosomas. Ejecutaremos estos algoritmos con un tamaño de población de 50 cromosomas, una probabilidad de cruce de 1 y una probabilidad de mutación del 0'001. Existirán dos variaciones, usando dos operadores de cruce distintos.
	\item Algoritmos Meméticos: Estos algoritmos no se basan totalmente en el esquema de evolución, lo combinaran con al búsqueda local, aplicándola cada cierto tiempo sobre todos o parte de los cromosomas, de forma que usará la exploración de los algoritmos genéticos para buscar nuevas zonas del espacio con la explotación de la búsqueda local para encontrar los mínimos de dichas zonas del espacio.  Ejecutaremos estos algoritmos con un tamaño de población de 10 cromosomas, una probabilidad de cruce de 1 y una probabilidad de mutación del 0'001. Existirán tres variaciones, aplicando la búsqueda local a todos los cromosomas, a un 10\% de ellos o al 10\% de los mejores cromosomas. Siempre aplicaremos la búsqueda local cada 10 generaciones.
\end{enumerate}

Como hemos comentado, los algoritmos genéticos son algoritmos que favorecerán la exploración, por lo que nos permitirán explorar en zonas donde los mínimos sean distintos, pero por el contrario no favorecerán tanto la explotación, por lo que, en teoría, les costará más encontrar el mínimo. El objetivo de está práctica es estudiar si esto ocurre y de que forma controlar el comportamiento de los algoritmos aplicando los distintos operadores.

\subsubsection{Representación para implementar estos algoritmos}

Como hemos comentado en la sección 2, la implementación de la práctica 1 no nos funcionará para aplicar los operadores.

En esta práctica usaremos la nueva representación, pero además, adelantando la condición de parada de los algoritmos, el número de evaluaciones será muy importante, por lo que al almacenar una población almacenaremos con cada cromosoma su valor de la función objetivo, y no tener que perder una evaluación cada vez que queramos conocerlo, haciendo que solo evaluemos cuando sea apliquemos los operadores que modifiquen un cromosoma.

\subsubsection{Generador población inicial.}

Todos nuestros algoritmos evolutivos comenzarán con una población inicial aleatoria, luego este operador nos ayudará a generarla:

\begin{lstlisting}
Devuelve vector de soluciones - Recibe tamaño de la poblacion tam_pob
	poblacion = vector de soluciones vacio
	
	Mientras poblacion.size() < tam_pob:
		poblacion.añadir(clusters_to_solucion(generar_solucion_aleatoria));
\end{lstlisting}

\subsubsection{Operador de selección.}

Este operador se encargará de realizar la selección de la población sobre la que aplicar los cruces y mutaciones. En esta práctica usaremos como operador de selección un torneo binario, es decir, sacaremos dos cromosomas aleatorios, y escogeremos el de mejor función objetivo (en nuestro caso, como queremos minimizar, el de menor función objetivo), cabe destacar que usando este método el cromosoma de peor calidad nunca se escogerá, ya que siempre perderá el torneo, y además es posible que aparezcan más de una copia del mismo cromosoma, haciendo que el mejor cromosoma al ganar siempre el torneo aparecerá con mayor frecuencia, consiguiendo con esto que se seleccione a los mejores cromosomas de la población. Este operador tendrá una pequeña diferencia dependiendo del algoritmo a usar:

\begin{itemize}
	\item Algoritmos Genéticos Generacionales y Algoritmos Meméticos: La población intermedia seleccionada tendrá tantos cromosomas como la población inicial
	\item Algoritmos Genéticos Estacionarios: La población intermedia seleccionada será únicamente de dos elementos.
\end{itemize}

Dicho esto observamos que el operador de selección será común para todos los algoritmos, únicamente variando el número de veces en las que aplicar el torneo binario. Con esto quedaría la siguiente implementación:

\begin{lstlisting}
Devuelve un vector de parejas <cromosoma, valoración> - recibe: vector de parejas <cromosoma, valoracion> pob_inicial, entero con el número de elementos a seleccionar tam

	pob_seleccionada = vector de parejas <cromosoma, valoracion> vacio
	
	Mientras pob_seleccionada.size() < tam:
		primer_candidato = aleatorio (0, pob_ini.size())
		
		// evitamos hacer el torneo entre un elemento y si mismo
		Hacer:
				segundo_candidato =  aleatorio (0, pob_ini.size())
		Mientas segundo_candidato == primer_candidato;
		
		Si pob_ini[primer_candidato].valoracion > poblacion[segunda_candidato].valoracion:
			//buscamos minimizar, si el primer candidato tiene valoracion mayor, el mejor es el segundo
			pob_seleccionada.añadir (pob_ini[segundo_candidato])
		Si no:
			pob_seleccionada.añadir(pob_ini[primer_candidato])


	Devolver pob_seleccionada
\end{lstlisting}

De esta forma conseguiremos seleccionar la población que formará parte del cruce y mutación.


\subsubsection{Operador de reparación.}

Uno de los problemas que tendremos es que cuando realicemos la operación de cruce es posible que se genere una solución infactible, esto lo vamos a solucionar aplicando un operador de reparación, que simplemente escoja un elemento de forma aleatoria y lo introduzca en el cluster vacío:

\begin{lstlisting}
Recibe un cromosoma solucion
	contador = vector de tamaño num_clusters
	
	// tenemos un contador que será un vector, donde nos dirá cuantos elementos 
	// hay en el cluster X si hacemos contador[X]
	Para i = 0 hasta i < solucion.size(); i++
		contador[solucion[i]]++
		
	Para i = 0; i < contador.size(); i++
		Si contador[i] == 0:
			// cluster vacio
			elemento
			Hacer:
				// sacamos aleatorios hasta que escojamos uno que si lo quitamos
				// de su cluster no deja un cluster vacio
				elemento = aleatorio(0, solucion.size()) 
			Mientras contador[solucion[elemento]] - 1 == 0
				
			// cambiamos el elemento aleatorio al cluster vacio
			contador[solucion[elemento]]--;
			solucion[elemento] = i
			contador[i]++		
\end{lstlisting}

\subsubsection{Operador de cruce uniforme.}

Esta será una de las dos formas de aplicar un cruce. Este operador recibirá una población y una probabilidad de cruce, y para evitar tener que generar una gran cantidad de números aleatorios aplicará la esperanza matemática para decidir si una pareja cruza o no. Las parejas de la población, al ya estar seleccionada de forma aleatoria, serán los indices pares con su siguiente indice, es decir, las parejas serán la \{0, 1\}, \{2, 3\}, etc.

 Este operador obtendrá de forma aleatoria N/2 indices del cromosoma aleatorios y sin repetición siendo N el número de genes de los cromosomas, de forma que el hijo generado mantendrá la asignación del primer padre de los elementos obtenidos de forma aleatoria, mientras que la asignación de los otros elementos serán del segundo padre. Como vemos este operador de cruce no da prioridad a ninguno de los dos padres.
 
Este operador recibirá por referencia la población y la probabilidad de cruce, y devolverá el número de evaluaciones realizadas. En este operador usaremos las herramientas explicadas en el cambio de representación, ya que evaluar la solución es mucho más fácil con la representación de la práctica 1.
 
La implementación sería la siguiente:

\begin{lstlisting}
Devuelve el número de evaluaciones - Recibe un vector de parejas <cromosoma, valoracion> pob y una probabilidad de cruce p_cruce

	NUM_PAREJAS = pob.size()/2
	// aplicamos la esperanza matemática
	NUM_CRUCES = NUM_PAREJAS * p_cruce
	
	evaluaciones = 0
	
	indice = 0
	
	// hacemos el doble de cruces ya que cada pareja tiene que generar dos hijos
	Para i = 0 hasta NUM_CRUCES*2; salto para i = +1:
		valores = {}

		// hasta que no generemos la mitad de indices del cromosoma		
		Mientras valores.size() < pob[indice].cromosoma.size()/2:
			valor = aleatorio(0, pob[indice].cromosoma.size())
			
			// si el valor no esta lo añadimos
			Si valores.find(valor ) == valores.end() 
				valores.añadir(valor)
				
		cruce = nuevo vector de tamaño pob[indice].cromosoma.size()
		
		Desde j = 0 hasta pob[indice].cromosoma.size():
			Si j esta en valores:
				cruce[j] = pob[indice].cromosoma[j]
			Si no:
				cruce[j] = pob[indice+1].cromosoma[j]
				
		reparar_cruce(cruce)
		
		Cargar solucion en clusters con solucion_to_clusters(cruce)
		
		hijo = {cruce, funcion_objetivo(cruce)}
		evaluaciones++
		
		pob.añadir(hijo)
		
		// como siempre cruzamos los dos primeros, cuando estamos en i = 1, ya hemos
		// generado y añadido al final los dos hijos de i = 0 e  i = 1, luego borramos los
		// dos primeros elementos de la población
		Si (i % 2 == 1 ):
			pob.erase(pob.begin())
			pob.erase(pob.begin())
					

	Devolver evaluaciones;

\end{lstlisting}

No devolvemos la nueva población ya que al ser una referencia se va modificando mientras se ejecuta el algoritmo.

\subsubsection{Operador de cruce de segmento fijo.}

Esta será la otra forma de aplicar un cruce. Los parámetros de entrada y la forma de organizar las parejas es la misma que en el operador de cruce uniforme.

Este operador obtendrá de forma aleatoria dos valores, uno será el inicio del segmento y otro el tamaño del segmento, este operador al generar un hijo copiará dicho segmento de un padre, y el resto de elementos se repartirán entre los dos padres de forma uniforme, tal como vimos en el cruce uniforme. Como vemos este método favorece al padre del que escojamos el segmento ya que pasará más información al hijo que el otro padre, por lo que tenemos dos opciones:

\begin{enumerate}
	\item Seleccionar al padre de mejor función objetivo: Favorecemos la explotación, al intentar mantener la información de una buena solución.
	\item Seleccionar al padre de peor función objetivo: Favorecemos la exploración ya que la solución tardará más en converger.
\end{enumerate}

Para desarrollar este operador he decidido favorecer la explotación, ya que de por si los algoritmos a implementar cuentan con gran exploración. 

 
Este operador recibirá por referencia la población y la probabilidad de cruce, y devolverá el número de evaluaciones realizadas. En este operador usaremos las herramientas explicadas en el cambio de representación, ya que evaluar la solución es mucho más fácil con la representación de la práctica 1.

 
La implementación sería la siguiente:

\begin{lstlisting}
Devuelve el número de evaluaciones - Recibe un vector de parejas <cromosoma, valoracion> pob y una probabilidad de cruce p_cruce

	NUM_PAREJAS = pob.size()/2
	// aplicamos la esperanza matemática
	NUM_CRUCES = NUM_PAREJAS * p_cruce
	
	evaluaciones = 0
	
	indice_p1 = 0
	indice_p2 = 0
	
	// hacemos el doble de cruces ya que cada pareja tiene que generar dos hijos	
	Para i = 0 hasta NUM_CRUCES*2; salto para i = +1:
	
		indice_p1 = 0
		indice_p2 = 1	
	
		Si pob[indice_p2].valoracion	< pob[indice_p1].valoracion:
			indice_p1 = 1
			indice_p2 = 0
	
	
		tam_segmento = aleatorio(0, pob[i].cromosoma.size())
		ini_segmento = aleatorio(0, pob[i].cromosoma.size())
		
		fin_segmento = (ini_segmento + tam_segmento + 1
	
		cruce = nuevo vector de tamaño pob[indice].cromosoma.size()
	
		// copiamos el segmento fijo
		Para j = ini_segmento; j <= ini_segmento + tam_segmento:
			cruce[j%pob[indice_p1].cromosoma.size()] = pob[indice_p1].cromosoma[j%pob[indice_p1].cromosoma.size()]
	
		rango_fijo_low = fin_segmento;

		Si fin_segmento >=pob[indice_p1].cromosoma.size()
			rango_fijo_low = (fin_segmento % pob[indice_p1].cromosoma.size());
			rango_fijo_hight = ini_segmento;
		Si no
			rango_fijo_low = fin_segmento;
			rango_fijo_hight = pob[indice_p1].cromosoma.size() + ini_segmento;

		valores.clear();
		valores = {}

		// hasta que no generemos la mitad de indices del cromosoma		
		Mientras valores.size() < (rango_fijo_hight-rango_fijo_low)/2:
			valor = aleatorio(rango_fijo_low, rango_fijo_hight)
			
			// si el valor no esta lo añadimos
			Si valores.find(valor ) == valores.end() 
				valores.añadir(valor % pob[indice_p1].cromosoma.size())
		
		Desde j = rango_fijo_low hasta rango_fijo_hight:
			val = j  % pob[indice_p1].cromosoma.size()
			Si v esta en valores:
				cruce[v] = pob[indice].cromosoma[v]
			Si no:
				cruce[v] = pob[indice+1].cromosoma[v]
				
		reparar_cruce(cruce)
		
		Cargar solucion en clusters con solucion_to_clusters(cruce)
		
		hijo = {cruce, funcion_objetivo(cruce)}
		evaluaciones++
		
		pob.añadir(hijo)
		
		// como siempre cruzamos los dos primeros, cuando estamos en i = 1, ya hemos
		// generado y añadido al final los dos hijos de i = 0 e  i = 1, luego borramos los
		// dos primeros elementos de la población
		Si (i % 2 == 1 ):
			pob.erase(pob.begin())
			pob.erase(pob.begin())
					

	Devolver evaluaciones;

\end{lstlisting}

No devolvemos la nueva población ya que al ser una referencia se va modificando mientras se ejecuta el algoritmo.


\subsubsection{Operador de mutación uniforme.}

Este operador se encargará de realizar las mutaciones en los cromosomas. Una mutación consiste en alterar el valor de un gen por otro al azar, esto ocurrirá con cierta probabilidad de mutación dada.

De nuevo aplicaremos la esperanza matemática para decidir cuantos genes mutan, sin embargo no la podremos aplicar para decidir a que valor mutan. Otra cosa a tener en cuenta es que los valores de la probabilidad de mutación son muy bajos, luego no podremos aplicar la esperanza matemática de forma directa sobre el algoritmo genético estacionario.

\begin{lstlisting}
Devuelve el número de evaluaciones - Recibe un vector de parejas <cromosoma, valoracion> pob, la probabilidad de mutación p_mut y el tipo de generacion tipo_generacion.

	contador = vector de enteros de tamaño num_clusters
	evaluaciones = 0
	
	// si es estacionario probamos sacando probabilidad y esperanza aplicada al cromosoma
	// si no, aplicamos esperanza matemática
	Si tipo_generacion == ESTACIONARIO:
		Para cada elemento i de la poblacion:
			aleatorio = aleatorio_float(0,1)
			Si aleatorio < p_mut * i.cromosoma.size():
				num_mutaciones++

	Si no:
		num_mutaciones = pob.size() * pob[0].cromosoma.size() * p_mut;
		
		
		
	Para i = 0 hasta num_mutaciones:
		// obtenemos el cromosoma a mutar de forma aleatoria
		indice_elemento = aleatorio (0, pob.size())
		elemento_poblacion = pob[indice_elemento]
		
		contador.clear()
		
		Para j = 0 hasta elemento_poblacion.size():
			contador[elemento_poblacion.cromosoma[j]]++
			j++
			
		// seleccionamos el gen a mutar, teniendo en cuenta de que
		// no genere solucion infactible
		Hacer:
			gen = aleatorio(0, elemento_poblacion.size())
		Hasta contador[poblacion.cromosoma[gen] - 1 == 0]
		
		destino = aleatorio(0, num_clusters)
		elemento_poblacion.cromosoma[gen] = destino
		
		solucion_to_clusters[elemento_poblacion.cromosoma]
		elemento_poblacion.valoracion = funcion_objetivo()
		evaluaciones++
		i++
	
	devolver evaluaciones
\end{lstlisting}

\subsubsection{Búsqueda local suave.}

Como hemos comentado, los algoritmos meméticos que vamos a implementar se basan en crear un algoritmo genético generacional y cada cierto número de generaciones aplicar una búsqueda local a los cromosomas. Esto presenta un problema y es que con el número de cromosomas con el que nos pide el guión de prácticas ejecutar es demasiado bajo, y aplicar la búsqueda local implementada en la práctica 1 haría que converja de forma muy rápida, por lo que vamos a aplicar una búsqueda local más suave, que solo recorrerá el cromosoma una vez y tendrá un límite de fallos permitidos.

El algoritmo modificará por referencia la solución inicial dada.

\begin{lstlisting}
Devuelve numero de evaluaciones - Recibe Pareja <solucion, valoracion> sol_ini, número de fallos permitidos max_fallos

	indices = vector vacio
	contador = vector vacio de tamaño num_clusters
	
	// creamos un vector con los indices y contamos en que cluster están los elementos
	Para i = 0 hasta sol_ini.solucion.size():
		indices.añadir(i)
		contador[sol_ini.solucion[i]]++
		i++
	
	random_shuffle(indices)
	
	fallos = 0
	mejora = true
	i = 0
	evaluaciones = 0
	
	
	Mientras (mejora O fallos	 < max_fallos) Y i < sol_ini.solucion.size():
		mejora = false
		mejor_cluster = -1
		val_mejor_cluster = sol_ini.valoracion
		sol_intermedia = sol_ini
		
		Para j = 0 mientras j < num_clusters:
			// si no voy a probar el cluster en el que esta ya, o si no genero
			// una solucion infactible
			Si sol_ini.solucion[indices[i]] != j Y contador[sol_ini.solucion[indices[i]] - 1 > 0]:
				sol_intermedia.solucion[indices[i]] = j
				cargar solucion con solucion_to_clusters(sol_intermedia.solucion)
				sol_intermedia.valoracion = funcion_objetivo()
				evaluaciones++
				
				Si sol_intermedia.valoracion < val_mejor_cluster:
					mejor_cluster = j
					val_mejor_cluster = sol_intermedia.valoracion
					mejora = true
					
		
		Si no hay mejora:
			fallos++
		Si no:
			contador[sol_ini.solucion[indices[i]]]--
			sol_ini.solucion[indices[i]] = mejor_cluster
			sol_ini.valoracion = val_mejor_cluster
			contador[mejor_cluster]++			
		i++

	Devolver evaluaciones
\end{lstlisting}


\subsubsection{Implementación.}

Como vemos, tenemos una serie de operadores que tenemos que aplicar a distintos algoritmos, e incluso hacer variaciones sobre cuales utilizar, etc, cuando realmente existen dos grandes diferencias:

\begin{enumerate}
	\item Los algoritmos genéticos generacionales se ejecutarán exactamente igual que los algoritmos genéticos estacionarios, solo que con una variación en el número de elementos a seleccionar.
	\item Los algoritmos meméticos son iguales que los algoritmos genéticos, solo que aplicando una búsqueda local.
\end{enumerate}

Mencionar que se nos pedía que escogieramos nosotros el operador de cruce para los algoritmos meméticos. Como he mencionado en el desarrollo de los algoritmos, el algoritmos de cruce uniforme favorece la explotación ya que seleccionará el segmento fijo del padre con mejor función objetivo, por esto, como el algoritmo memético también explotará al aplicar la búsqueda local suave he decidido escoger el operador de cruce uniforme.

Por estos motivos he decidido implementar todos los algoritmos en una única función, que dependiendo de los parámetros de entrada aplicará unos operadores u otros.

Esto lo he conseguido definiendo los siguientes enumerados:

\begin{lstlisting}
enum class operador_cruce {SEGMENTO_FIJO, UNIFORME};
enum class tipo_generacion {GENERACIONAL, ESTACIONARIO, MEMETICO_1, MEMETICO_0_1, MEMETICO_0_1_MEJ};
\end{lstlisting}

\newpage

La implementación es la siguiente:

\begin{lstlisting}
Devuelve una pareja <solucion, valoracion>
Recibe: Evaluaciones máximas evaluaciones_max, 
		Tamaño de la poblacion inicial tam_pob_ini,
		Probabilidad de mutación prob_mutacion,
		Probabilidad de cruce prob_cruce, 
		Tipo de cruce tipo_cruce,
		Tipo de generacion tipo_gen,
		Si existe elitismo elitismo = true
	
	p = generar_poblacion_inicial(tam_pob_ini)
	poblacion = vector de parejas <cromosoma, valoracion> vacio
	
	Para cada elemento i de p:
		cargar solucion i a los clusters
		poblacion.añadir( i, funcion_objetivo() )
		
	poblacion_anterior = poblacion
	mejor = poblacion[0]
	
	Para cada elemento i de poblacion:
		Si i.valoracion < mejor.valoracion
			mejor = i		
	evaluaciones = 0
	Mientras evaluaciones < evaluaciones_max:
		// aplicamos la seleccion
		Si tipo_gen == ESTACIONARIO
			poblacion = seleccion_algoritmos_ev(poblacion_anterior, 2)
		Si no
			poblacion = seleccion_algoritmos_ev(poblacion_anterior, poblacion_anterior.size())		
			
		Si tipo_cruce == SEGMENTO_FIJO
			evaluaciones += operador_cruce_seg_fijo(poblacion, prob_cruce)
		Si no 
			evaluaciones += operador_cruce_uniforme(poblacion, prob_cruce)		
			
		evaluaciones += operador_mutacion_uniforme(poblacion, prob_mutacion, tipo_generacion)
		Si tipo_gen != ESTACIONARIO:
			// si tenemos elitismo, buscamos si esta el mejor, si no está el mejor
			// buscamos el peor e introducimos el mejor de la poblacion anterior
			Si elitismo:
				pos_mejor = poblacion.find(mejor)
				
				// si no está
				Si pos_mejor == poblacion.end():
					Buscamos el peor elemento
					poblacion.borrar(peor_elemento)
					poblacion.añadir(mejor)			
			poblacion_anterior = poblacion
		Si no
			Buscamos los dos peores elementos peor_1 y peor_2 donde peor_1 es el peor de todos en la poblacion anterior
			//sabemos que la poblacion tiene dos elementos
			Mejor 1 es el mejor de poblacion
			Mejor 2 es el segundo mejor de la poblacion 2
			
			// basicamente sustituir los dos mejores por los dos peores en orden
			Si mejor_2 es mejor que peor 1:
				sustituir peor_1 por mejor_2 en poblacion_anterior
				sustituir peor_2 por mejor_1 en poblacion_anterior
			Si no:
				Si mejor_2 es mejor que peor_2:
					sustituir mejor_2 por peor_2 en poblacion_anterior
				Si no
					Si mejor_1 es mejor que peor 1:
						sustituir mejor_1 por peor_1 en poblacion_anterior		
					Si no
						Si mejor_1 es mejor que peor 2:
							sustituir mejor_1 por peor_2 en poblacion_anterior		
						
		// en caso de que sea memetico aplicamos la BLS
		// el guión de prácticas nos dice que el número de fallos permitidos es
		// 0.1 * tamaño de la solucion
		Si generacion % poblacion_anterior.size() == 0:
			Si tipo_gen == MEMETICO_1:
				Para todos los elementos i de poblacion_anterior:
					evaluaciones += algoritmo_BL_suave(i, 0.1*i.solucion.size())
			Si tipo_gen == MEMETICO_0_1
				// aplicamos esperanza matemática
				num_bl = 0.1 * poblacion_anterior,size()
				Para i = 0 hasta num_bl:
					evaluaciones += algoritmo_BL_suave(i, 0.1*i.solucion.size())
			Si tipo_gen == MEMETICO_0.1_MEJ
				num_bl = 0.1 * poblacion_anterior,size()
				Buscamos los num_bl mejores cromosomas y los almacenamos en un  set de indices
				Para todos los elementos i del set de indices:
					evaluaciones += algoritmo_BL_suave(poblacion_anterior[i], 0.1*poblacion_anterior[i].solucion.size())
		
			
		Para cada elemento i de poblacion:
			Si i.valoracion < mejor.valoracion
				mejor = i

		generacion++
		
	cargar solucion en clusters
	
	Devolver {solucion, funcion_objetivo()}

	
		
\end{lstlisting}

Notar que en mi implementación la evaluación de la población inicial no eleva el contador de evaluaciones ya que lo he contado como inicialización del algoritmo, aun así la variación sería unas 50 evaluaciones a lo sumo que no harían variar la ejecución. Además, aunque superemos el límite de evaluaciones nos aseguramos de finalizar un ciclo del esquema de evolución y no dejar una población intermedia sin finalizar.

Con esta implementación podemos ejecutar todos los algoritmos desde este método de nuestro problema, de esta forma también estamos reflejando que en realidad los algoritmos evolutivos son una cantidad enorme de algoritmos, que existen infinidad de variaciones, ajustes, etc, pero sin embargo la idea es la misma y siempre bajo el mismo esquema de evolución.

\newpage

\section{Algoritmo de comparación.}

Como algoritmo de comparación usaremos un algoritmo Greedy, basado en una variación del algoritmo k-medias.

\subsection{Algoritmo Greedy.}

El algoritmo se basa en recorrer los indices de forma aleatoria, asignando los elementos al cluster que menos restricciones incumpla, y de entre los que cumplan esta condición al más cercano. Este algoritmo se centrará en tener el menor número de restricciones, aunque la desviación general sea mucho mayor.

\subsubsection{Función para asignar un cluster}

Para resolver el problema de, una vez escogemos un elemento, buscar y asignar un cluster a dicho elemento he diseñado esta función. El principal cometido es calcular el incremento de la \textit{infeasibility} asociada a introducir el elemento en los distintos clusters usando la función para calcular la \textit{infeasibility} parcial antes comentada. Una vez tenemos los clusters que menos restricciones incumplen nos quedamos con el que tenga el centroide a menor distancia y lo asignamos a dicho cluster.


\begin{lstlisting}
elemento = pasado como argumento
distancia = 0
menor_distancia = infinito
menor_restricciones = infinito

// usamos un vector de pares para saber el indice al ordenarlos
Para cada indice i desde 0 hasta clusters.size()
	aumento_infactibilidad.push_back({i, incumple_restricciones(elemento, clusters[i])})

sort(aumento_infactibilidad)
menor_restricciones = aumento_infactibilidad[0].second

Para cada indice i desde 0 hasta aumento_infactibilidad.size() Y aumento_infactibilidad[i].second == menor_restricciones
	distancia = distancia_puntos(clusters[aumento_infactibilidad[i].first.get_centroide(), datos[elemento]])
	Si distancia < menor_distancia
		menor_distancia = distancia
		cluster_menor_distancia = aumento_infactibilidad[i].first


return cluster_menor_distancia
\end{lstlisting}


\subsubsection{Desarrollo del algoritmo.}

Con el uso de esta función y las descritas en la sección común podemos desarrollar el algoritmo de la siguiente forma:

\begin{lstlisting}
limpiamos los clusters
inicializamos los centroides de forma aleatoria

indices = {0 ... datos.size()}

random_shuffle(indices)

hay_cambios = false
cambios = vector de booleanos con tamaño clusters.size() inicializado todo a false

sol_antigua = clusters

do:
	hay_cambios = false
	cambios = {false, ..., false}
	
	Para cada elemento i de indices:
		num_cluster = buscar_cluster(i)
		clusters[num_cluster].add_elemento(i)
		
	Para cada indice i de 0 hasta clusters.size()
		cambios[i] = sol_antigua[i].get_elementos != clusters[i].get_elementos()
		
		Si cambios[i]
			clusters[i].calcular_centroide();
			sol_antigua[i] = clusters[i];
			
		clusters[i].limpiar()
		
	Para cada indice i de 0 hasta cambios.size()
		hay_cambios = hay_cambios || cambios[i]



while hay_cambios

calcular_desviacion_general()
\end{lstlisting}

Notar que está factorizado, de forma que solo se actualicen los centroides que han modificado sus elementos con respecto a la iteración anterior, sin embargo para volver a iterar tenemos que comprobar que al menos un cluster ha cambiado, de ahí actualizar \texttt{hay\_cambios} en función de todos los clusters.

Como hemos comentado, este algoritmo greedy prioriza el minimizar el número de restricciones incumplidas, y entre los valores que menos incumplen, el cluster más cercano, por lo que en un principio podemos pensar que conseguiremos una baja \textit{infeasibility}, sin embargo, como la decisión de introducir un elemento en un cluster influye de cara a los próximos elementos, y al no tener en cuenta próximos elementos ni ser capaces de volver atrás, esto hará que no consigamos esas \textit{infeasibility} tan baja como creíamos.

\section{Proceso de implementación.}

Para la implementación he desarrollado mis propias clases en C++, como adjunto en la carpeta de fuentes. Para las estructuras de datos he utilizado la STL:

\begin{itemize}
	\item Clase map para las restricciones.
	\item Clase vector para los datos y centroides.
	\item Clase set para los elementos de los clusters.
	\item Clase pair para almacenar las parejas de restricciones, así como para operaciones auxiliares, como el uso en la función auxiliar del algoritmo greedy o la población en algoritmos genéticos.
\end{itemize}

Además de estas clases de C++ y las implementadas por mi explicadas a lo largo de este guión también he usado las funciones de generación de números aleatorios dadas por los profesores añadiendo algunas funciones, como por ejemplo generar un número aleatorio entre 0 y un número dado, de forma que el número generado N cumpla que 0 < N entre otras.

Para medir el tiempo de ejecución he utilizado las funciones dadas por los profesores de la asignatura.


\subsection{Manual de uso.}

En la carpeta del código fuente existe un fichero Markdown (se puede abrir como texto plano, pero recomiendo un lector de Markdown para facilitar la lectura, editores como Atom o VSCode tienen uno integrado) README.md en el que se explica toda la estructura del código, carpetas, etc.

En esta sección haré un resumen de esto con lo necesario para compilar y ejecutar el programa.

\subsubsection{Compilar:}

Para compilar el programa hay que moverse a la carpeta de fuentes, donde se encuentra el archivo \texttt{Makefile} y ejecutar:

\begin{lstlisting}
make
\end{lstlisting}

Esto nos generará en la carpeta \textit{bin/} el ejecutable.


\subsubsection{Ejecución:}

Podemos lanzar el programa con:

\begin{lstlisting}
./bin/practica1 <fichero_datos> <fichero_restricciones> <num_clusters> <semilla>
\end{lstlisting}

Cada ejecución del programa lanzará todos los algoritmos explicados en esa misma ejecución.

También cabe destacar que en la ruta del fichero de datos se creará un fichero con la extensión \texttt{.out} con la salida y solución del PAR, este fichero contendrá en su nombre la semilla con la que se ha ejecutado, el algoritmo y el conjunto de datos usado.

En la carpeta de gráficas encontraremos una serie de ficheros .gp y dos carpetas:

\begin{itemize}
	\item Carpeta datos: El programa automáticamente generará estos ficheros que usaremos de entrada para generar los gráficos.
	\item Carpeta salida\_png: Donde se generarán los gráficos.
\end{itemize}

Para generar los gráficos simplemente tenemos que ejecutar el programa y despues ejecutar \texttt{gnuplot <fichero>.gp} dentro de la carpeta gráficas para obtener los png.

Estas gráficas contendrán el avance de los algoritmos genéticos con respecto a las generaciones.

\newpage

\section{Experimentos y análisis de resultados.}


\subsection{Descripción de los casos.}

En nuestro caso, para el problema del PAR trabajaremos sobre tres conjuntos de datos:

\begin{itemize}
	\item{Iris: Conjunto de datos sobre tres tipos de flores Iris. En este caso tendremos un conjunto de 150 datos y el objetivo será clasificar estas flores según su tipo.}
	\item{Ecoli: Conjunto de datos con características de células, empleadas para predecir la localización de proteínas. En total son 336 datos, de 8 clases distintas.}
	\item{Rand: Conjunto de datos artificial, formado por tres conjuntos de datos bien diferenciados generados a base de distribuciones normales. En total 150 datos con 3 clasificaciones distintas.}
	\item{Newthyroid: Conjunto de datos con medidas cuantitativas tomadas sobre la glándula tiroides de 215 pacientes con 3 clasificaciones distintas.} 
\end{itemize}

Destacar que en nuestro problema las restricciones tendrán un papel muy importante, ya que los conjuntos de datos son muy distintos y estas restricciones son un añadido no original del problema, podría darse el caso de que dos elementos que en principio son de la misma clase estén separados por las restricciones, en caso de tener muy en cuenta las restricciones en nuestra función objetivo, o viceversa, es decir, que realice la asignación de clases sin tener en cuenta las restricciones si a estas no se les da la suficiente importancia.


\subsubsection{Semillas escogidas.}

Las semillas que voy a usar son:

\begin{itemize}
	\item {123452244}
	\item {9398429}
	\item {12321}
	\item {213566}
	\item {3939021}
\end{itemize}

\subsection{Teorema No Free Lunch}

Como hemos visto en teoría, las metaheurísticas son métodos de resolución genéricos que podemos aplicar a cualquier problema, y siguiendo el teorema de no free lunch, una metaheurística que se comporte bien en un tipo de problemas se comportará mal en el resto de problemas, luego los análisis que realizaré en este documento se centrarán exclusivamente en el Problema de Asignación con Restricciones, esto quiere decir que es muy probable que las metaheurísticas más adelante analizadas se comporten de manera distinta sobre otros problemas y por tanto este análisis no se podrá extrapolar a otros problemas.

\newpage


\subsection{Resultados obtenidos.}

Estas son las tablas con los resultados de las distintas semillas:



\subsubsection{AGG con operador de cruce uniforme.}

% Please add the following required packages to your document preamble:
% \usepackage{multirow}
\begin{table}[H]
\footnotesize
\begin{tabular}{|c|c|c|c|c|c|c|c|c|}
\hline
\multicolumn{9}{|c|}{\textbf{Resultados obtenidos por el algoritmo AGG-UN en el PAR con 10\% de restricciones}}                                                                                                   \\ \hline
\multirow{2}{*}{} & \multicolumn{4}{c|}{\textbf{Iris}}                                                            & \multicolumn{4}{c|}{\textbf{Ecoli}}                                                           \\ \cline{2-9} 
                  & \textit{\textbf{Tasa\_C}} & \textit{\textbf{Tasa\_inf}} & \textit{\textbf{Agr,}} & \textbf{T} & \textit{\textbf{Tasa\_C}} & \textit{\textbf{Tasa\_inf}} & \textit{\textbf{Agr,}} & \textbf{T} \\ \hline
123452244         & 0,595316                  & 0                           & 0,595316               & 35,2706    & 1142,03                   & 65                          & 1405,36                & 268,878    \\ \hline
9398429           & 0,595316                  & 0                           & 0,595316               & 37,548     & 1107,2                    & 76                          & 1415,09                & 265,308    \\ \hline
12321             & 0,595316                  & 0                           & 0,595316               & 31,2654    & 1193,6                    & 123                         & 1691,89                & 146,331    \\ \hline
213566            & 0,595316                  & 0                           & 0,595316               & 32,6679    & 1097,62                   & 62                          & 1348,79                & 152,271    \\ \hline
3939021           & 0,595316                  & 0                           & 0,595316               & 32,784     & 1166,27                   & 102                         & 1579,49                & 155,038    \\ \hline
\textbf{Media}    & 0,595316                  & 0                           & 0,595316               & 33,90718   & 1141,344                  & 85,6                        & 1488,124               & 197,5652   \\ \hline
\end{tabular}
\end{table}

% Please add the following required packages to your document preamble:
% \usepackage{multirow}
\begin{table}[H]
\footnotesize

\begin{tabular}{|c|c|c|c|c|c|c|c|c|}
\hline
\multicolumn{9}{|c|}{\textbf{Resultados obtenidos por el algoritmo AGG-UN en el PAR con 10\% de restricciones}}                                                                                                   \\ \hline
\multirow{2}{*}{} & \multicolumn{4}{c|}{\textbf{Rand}}                                                            & \multicolumn{4}{c|}{\textbf{Newthyroid}}                                                      \\ \cline{2-9} 
                  & \textit{\textbf{Tasa\_C}} & \textit{\textbf{Tasa\_inf}} & \textit{\textbf{Agr,}} & \textbf{T} & \textit{\textbf{Tasa\_C}} & \textit{\textbf{Tasa\_inf}} & \textit{\textbf{Agr,}} & \textbf{T} \\ \hline
123452244         & 0,746683                  & 0                           & 0,746683               & 36,1749    & 288,636                   & 6                           & 307,093                & 73,114     \\ \hline
9398429           & 0,746683                  & 0                           & 0,746683               & 37,1932    & 301,551                   & 86                          & 566,1                  & 69,8764    \\ \hline
12321             & 0,746683                  & 0                           & 0,746683               & 32,1782    & 288,636                   & 6                           & 307,093                & 72,5568    \\ \hline
213566            & 0,746683                  & 0                           & 0,746683               & 30,2733    & 288,636                   & 6                           & 307,093                & 67,1617    \\ \hline
3939021           & 0,746683                  & 0                           & 0,746683               & 30,2733    & 288,636                   & 6                           & 307,093                & 66,7465    \\ \hline
\textbf{Media}    & 0,746683                  & 0                           & 0,746683               & 32,4795    & 291,219                   & 22                          & 358,8944               & 69,89108   \\ \hline
\end{tabular}
\end{table}


% Please add the following required packages to your document preamble:
% \usepackage{multirow}
\begin{table}[H]
\footnotesize

\begin{tabular}{|c|c|c|c|c|c|c|c|c|}
\hline
\multicolumn{9}{|c|}{\textbf{Resultados obtenidos por el algoritmo AGG-UN en el PAR con 20\% de restricciones}}                                                                                                   \\ \hline
\multirow{2}{*}{} & \multicolumn{4}{c|}{\textbf{Iris}}                                                            & \multicolumn{4}{c|}{\textbf{Ecoli}}                                                           \\ \cline{2-9} 
                  & \textit{\textbf{Tasa\_C}} & \textit{\textbf{Tasa\_inf}} & \textit{\textbf{Agr,}} & \textbf{T} & \textit{\textbf{Tasa\_C}} & \textit{\textbf{Tasa\_inf}} & \textit{\textbf{Agr,}} & \textbf{T} \\ \hline
123452244         & 0,595316                  & 0                           & 0,595316               & 65,2196    & 1016,71                   & 146                         & 1312,44                & 452,349    \\ \hline
9398429           & 0,595316                  & 0                           & 0,595316               & 63,107     & 1052,33                   & 89                          & 1232,61                & 438,982    \\ \hline
12321             & 0,595316                  & 0                           & 0,595316               & 58,3115    & 1091,82                   & 63                          & 1219,43                & 246,659    \\ \hline
213566            & 0,595316                  & 0                           & 0,595316               & 60,5067    & 986,45                    & 96                          & 1180,91                & 246,645    \\ \hline
3939021           & 0,595316                  & 0                           & 0,595316               & 54,3243    & 1019,08                   & 102                         & 1225,69                & 250,075    \\ \hline
\textbf{Media}    & 0,595316                  & 0                           & 0,595316               & 60,29382   & 1033,278                  & 99,2                        & 1234,216               & 326,942    \\ \hline
\end{tabular}
\end{table}


% Please add the following required packages to your document preamble:
% \usepackage{multirow}
\begin{table}[H]
\footnotesize

\begin{tabular}{|c|c|c|c|c|c|c|c|c|}
\hline
\multicolumn{9}{|c|}{\textbf{Resultados obtenidos por el algoritmo AGG-UN en el PAR con 20\% de restricciones}}                                                                                                   \\ \hline
\multirow{2}{*}{} & \multicolumn{4}{c|}{\textbf{Rand}}                                                            & \multicolumn{4}{c|}{\textbf{Newthyroid}}                                                      \\ \cline{2-9} 
                  & \textit{\textbf{Tasa\_C}} & \textit{\textbf{Tasa\_inf}} & \textit{\textbf{Agr,}} & \textbf{T} & \textit{\textbf{Tasa\_C}} & \textit{\textbf{Tasa\_inf}} & \textit{\textbf{Agr,}} & \textbf{T} \\ \hline
123452244         & 0,746683                  & 0                           & 0,746683               & 63,3179    & 313,299                   & 0                           & 313,299                & 125,794    \\ \hline
9398429           & 0,746683                  & 0                           & 0,746683               & 65,7837    & 313,299                   & 0                           & 313,299                & 131,859    \\ \hline
12321             & 0,746683                  & 0                           & 0,746683               & 60,6053    & 313,299                   & 0                           & 313,299                & 107,024    \\ \hline
213566            & 0,746683                  & 0                           & 0,746683               & 57,2568    & 313,299                   & 0                           & 313,299                & 83,0377    \\ \hline
3939021           & 0,746683                  & 0                           & 0,746683               & 48,94      & 313,299                   & 0                           & 313,299                & 80,4087    \\ \hline
\textbf{Media}    & 0,746683                  & 0                           & 0,746683               & 59,18074   & 313,299                   & 0                           & 313,299                & 105,62468  \\ \hline
\end{tabular}
\end{table}


\newpage


Usando el algoritmo genético generacional con cruce uniforme tanto en Iris como en Rand nos converge al mismo mínimo en todas las ejecuciones. En el conjunto de datos Newthyroid vemos como usando un 10\% de restricciones existen dos mínimos, el mejor mínimo lo alcanza cuatro veces y el peor de una única vez, este conjunto de datos, más complejo que rand e iris pero menos que ecoli, logra sin embargo obtener un buen mínimo sin incumplir restricciones con un 20\% de restricciones. En ecoli vemos como las soluciones están en torno a los 1300/1400 de agregado, que más adelante compararemos con otros algoritmos.

\subsubsection{AGG con operador de cruce de segmento fijo.}

% Please add the following required packages to your document preamble:
% \usepackage{multirow}
\begin{table}[H]
\footnotesize
\begin{tabular}{|c|c|c|c|c|c|c|c|c|}
\hline
\multicolumn{9}{|c|}{\textbf{Resultados obtenidos por el algoritmo AGG-SF en el PAR con 10\% de restricciones}}                                                                                                   \\ \hline
\multirow{2}{*}{} & \multicolumn{4}{c|}{\textbf{Iris}}                                                            & \multicolumn{4}{c|}{\textbf{Ecoli}}                                                           \\ \cline{2-9} 
                  & \textit{\textbf{Tasa\_C}} & \textit{\textbf{Tasa\_inf}} & \textit{\textbf{Agr,}} & \textbf{T} & \textit{\textbf{Tasa\_C}} & \textit{\textbf{Tasa\_inf}} & \textit{\textbf{Agr,}} & \textbf{T} \\ \hline
123452244         & 0,595316                  & 0                           & 0,595316               & 30,7132    & 1055,24                   & 45                          & 1237,54                & 228,143    \\ \hline
9398429           & 0,595316                  & 0                           & 0,595316               & 32,6323    & 1163,48                   & 103                         & 1580,75                & 227,444    \\ \hline
12321             & 0,595316                  & 0                           & 0,595316               & 32,8518    & 1348,35                   & 125                         & 1854,75                & 210,735    \\ \hline
213566            & 0,595316                  & 0                           & 0,595316               & 32,3785    & 1155,87                   & 77                          & 1467,81                & 211,249    \\ \hline
3939021           & 0,595316                  & 0                           & 0,595316               & 32,5782    & 1086,18                   & 46                          & 1272,54                & 216,735    \\ \hline
\textbf{Media}    & 0,595316                  & 0                           & 0,595316               & 32,2308    & 1161,824                  & 79,2                        & 1482,678               & 218,8612   \\ \hline
\end{tabular}
\end{table}

% Please add the following required packages to your document preamble:
% \usepackage{multirow}
\begin{table}[H]
\footnotesize
\begin{tabular}{|c|c|c|c|c|c|c|c|c|}
\hline
\multicolumn{9}{|c|}{\textbf{Resultados obtenidos por el algoritmo AGG-SF en el PAR con 10\% de restricciones}}                                                                                                   \\ \hline
\multirow{2}{*}{} & \multicolumn{4}{c|}{\textbf{Rand}}                                                            & \multicolumn{4}{c|}{\textbf{Newthyroid}}                                                      \\ \cline{2-9} 
                  & \textit{\textbf{Tasa\_C}} & \textit{\textbf{Tasa\_inf}} & \textit{\textbf{Agr,}} & \textbf{T} & \textit{\textbf{Tasa\_C}} & \textit{\textbf{Tasa\_inf}} & \textit{\textbf{Agr,}} & \textbf{T} \\ \hline
123452244         & 0,746683                  & 0                           & 0,746683               & 32,8901    & 288,636                   & 6                           & 307,093                & 70,1843    \\ \hline
9398429           & 0,746683                  & 0                           & 0,746683               & 31,6602    & 288,636                   & 6                           & 307,093                & 72,1044    \\ \hline
12321             & 0,746683                  & 0                           & 0,746683               & 33,2177    & 288,636                   & 6                           & 307,093                & 73,0704    \\ \hline
213566            & 0,746683                  & 0                           & 0,746683               & 32,6515    & 301,551                   & 86                          & 566,1                  & 73,9682    \\ \hline
3939021           & 0,746683                  & 0                           & 0,746683               & 32,6996    & 288,636                   & 6                           & 307,093                & 71,4673    \\ \hline
\textbf{Media}    & 0,746683                  & 0                           & 0,746683               & 32,62382   & 291,219                   & 22                          & 358,8944               & 72,15892   \\ \hline
\end{tabular}
\end{table}

% Please add the following required packages to your document preamble:
% \usepackage{multirow}
\begin{table}[H]
\footnotesize
\begin{tabular}{|c|c|c|c|c|c|c|c|c|}
\hline
\multicolumn{9}{|c|}{\textbf{Resultados obtenidos por el algoritmo AGG-SF en el PAR con 20\% de restricciones}}                                                                                                   \\ \hline
\multirow{2}{*}{} & \multicolumn{4}{c|}{\textbf{Iris}}                                                            & \multicolumn{4}{c|}{\textbf{Ecoli}}                                                           \\ \cline{2-9} 
                  & \textit{\textbf{Tasa\_C}} & \textit{\textbf{Tasa\_inf}} & \textit{\textbf{Agr,}} & \textbf{T} & \textit{\textbf{Tasa\_C}} & \textit{\textbf{Tasa\_inf}} & \textit{\textbf{Agr,}} & \textbf{T} \\ \hline
123452244         & 0,595316                  & 0                           & 0,595316               & 60,9859    & 1053,01                   & 117                         & 1290,01                & 328,32     \\ \hline
9398429           & 0,595316                  & 0                           & 0,595316               & 60,5222    & 1008,54                   & 110                         & 1231,35                & 344,946    \\ \hline
12321             & 0,595316                  & 0                           & 0,595316               & 58,6525    & 1017,47                   & 127                         & 1274,72                & 314,935    \\ \hline
213566            & 0,595316                  & 0                           & 0,595316               & 61,5564    & 1064,83                   & 77                          & 1220,8                 & 324,725    \\ \hline
3939021           & 0,595316                  & 0                           & 0,595316               & 59,312     & 1033,47                   & 116                         & 1268,44                & 326,822    \\ \hline
\textbf{Media}    & 0,595316                  & 0                           & 0,595316               & 60,2058    & 1035,464                  & 109,4                       & 1257,064               & 327,9496   \\ \hline
\end{tabular}
\end{table}

% Please add the following required packages to your document preamble:
% \usepackage{multirow}
\begin{table}[H]
\footnotesize
\begin{tabular}{|c|c|c|c|c|c|c|c|c|}
\hline
\multicolumn{9}{|c|}{\textbf{Resultados obtenidos por el algoritmo AGG-SF en el PAR con 20\% de restricciones}}                                                                                                   \\ \hline
\multirow{2}{*}{} & \multicolumn{4}{c|}{\textbf{Rand}}                                                            & \multicolumn{4}{c|}{\textbf{Newthyroid}}                                                      \\ \cline{2-9} 
                  & \textit{\textbf{Tasa\_C}} & \textit{\textbf{Tasa\_inf}} & \textit{\textbf{Agr,}} & \textbf{T} & \textit{\textbf{Tasa\_C}} & \textit{\textbf{Tasa\_inf}} & \textit{\textbf{Agr,}} & \textbf{T} \\ \hline
123452244         & 0,746683                  & 0                           & 0,746683               & 60,6225    & 313,299                   & 0                           & 313,299                & 124,651    \\ \hline
9398429           & 0,746683                  & 0                           & 0,746683               & 59,8896    & 313,299                   & 0                           & 313,299                & 128,853    \\ \hline
12321             & 0,746683                  & 0                           & 0,746683               & 62,5462    & 313,299                   & 0                           & 313,299                & 124,111    \\ \hline
213566            & 0,746683                  & 0                           & 0,746683               & 63,0179    & 313,299                   & 0                           & 313,299                & 122,92     \\ \hline
3939021           & 0,746683                  & 0                           & 0,746683               & 58,8071    & 313,299                   & 0                           & 313,299                & 112,465    \\ \hline
\textbf{Media}    & 0,746683                  & 0                           & 0,746683               & 60,97666   & 313,299                   & 0                           & 313,299                & 122,6      \\ \hline
\end{tabular}
\end{table}

En este caso vuelve a ocurrir lo mismo en los distintos conjuntos de datos, tanto iris como rand convergen al mismo punto, mientras que newthyroid tiene un punto mínimo común, aunque puede quedarse en otros mínimos peores, con ecoli seguimos encontrando mínimos más diversos debido a la complejidad del conjunto de datos y las características de sus elementos.

\subsubsection{AGE con operador de cruce uniforme.}

% Please add the following required packages to your document preamble:
% \usepackage{multirow}
\begin{table}[H]
\footnotesize
\begin{tabular}{|c|c|c|c|c|c|c|c|c|}
\hline
\multicolumn{9}{|c|}{\textbf{Resultados obtenidos por el algoritmo AGE-UN en el PAR con 10\% de restricciones}}                                                                                                   \\ \hline
\multirow{2}{*}{} & \multicolumn{4}{c|}{\textbf{Iris}}                                                            & \multicolumn{4}{c|}{\textbf{Ecoli}}                                                           \\ \cline{2-9} 
                  & \textit{\textbf{Tasa\_C}} & \textit{\textbf{Tasa\_inf}} & \textit{\textbf{Agr,}} & \textbf{T} & \textit{\textbf{Tasa\_C}} & \textit{\textbf{Tasa\_inf}} & \textit{\textbf{Agr,}} & \textbf{T} \\ \hline
123452244         & 0,595316                  & 0                           & 0,595316               & 38,2906    & 1114,98                   & 71                          & 1402,62                & 266,76     \\ \hline
9398429           & 0,595316                  & 0                           & 0,595316               & 37,2351    & 1025,71                   & 51                          & 1232,32                & 245,287    \\ \hline
12321             & 0,595316                  & 0                           & 0,595316               & 32,6881    & 1007,33                   & 44                          & 1185,58                & 148,58     \\ \hline
213566            & 0,595316                  & 0                           & 0,595316               & 32,7834    & 1058,01                   & 39                          & 1216,01                & 144,3      \\ \hline
3939021           & 0,595316                  & 0                           & 0,595316               & 31,9797    & 1066,17                   & 27                          & 1175,55                & 146,386    \\ \hline
\textbf{Media}    & 0,595316                  & 0                           & 0,595316               & 34,59538   & 1054,44                   & 46,4                        & 1242,416               & 190,2626   \\ \hline
\end{tabular}
\end{table}

% Please add the following required packages to your document preamble:
% \usepackage{multirow}
\begin{table}[H]
\footnotesize
\begin{tabular}{|c|c|c|c|c|c|c|c|c|}
\hline
\multicolumn{9}{|c|}{\textbf{Resultados obtenidos por el algoritmo AGE-UN en el PAR con 10\% de restricciones}}                                                                                                   \\ \hline
\multirow{2}{*}{} & \multicolumn{4}{c|}{\textbf{Rand}}                                                            & \multicolumn{4}{c|}{\textbf{Newthyroid}}                                                      \\ \cline{2-9} 
                  & \textit{\textbf{Tasa\_C}} & \textit{\textbf{Tasa\_inf}} & \textit{\textbf{Agr,}} & \textbf{T} & \textit{\textbf{Tasa\_C}} & \textit{\textbf{Tasa\_inf}} & \textit{\textbf{Agr,}} & \textbf{T} \\ \hline
123452244         & 0,746683                  & 0                           & 0,746683               & 37,8852    & 288,636                   & 6                           & 307,093                & 71,5246    \\ \hline
9398429           & 0,746683                  & 0                           & 0,746683               & 36,2917    & 288,636                   & 6                           & 307,093                & 80,6133    \\ \hline
12321             & 0,746683                  & 0                           & 0,746683               & 31,8545    & 288,636                   & 6                           & 307,093                & 47,9125    \\ \hline
213566            & 0,746683                  & 0                           & 0,746683               & 32,4578    & 288,636                   & 6                           & 307,093                & 66,8237    \\ \hline
3939021           & 0,746683                  & 0                           & 0,746683               & 30,3836    & 288,636                   & 6                           & 307,093                & 53,4953    \\ \hline
\textbf{Media}    & 0,746683                  & 0                           & 0,746683               & 33,77456   & 288,636                   & 6                           & 307,093                & 64,07388   \\ \hline
\end{tabular}
\end{table}


% Please add the following required packages to your document preamble:
% \usepackage{multirow}
\begin{table}[H]
\footnotesize
\begin{tabular}{|c|c|c|c|c|c|c|c|c|}
\hline
\multicolumn{9}{|c|}{\textbf{Resultados obtenidos por el algoritmo AGE-UN en el PAR con 20\% de restricciones}}                                                                                                   \\ \hline
\multirow{2}{*}{} & \multicolumn{4}{c|}{\textbf{Iris}}                                                            & \multicolumn{4}{c|}{\textbf{Ecoli}}                                                           \\ \cline{2-9} 
                  & \textit{\textbf{Tasa\_C}} & \textit{\textbf{Tasa\_inf}} & \textit{\textbf{Agr,}} & \textbf{T} & \textit{\textbf{Tasa\_C}} & \textit{\textbf{Tasa\_inf}} & \textit{\textbf{Agr,}} & \textbf{T} \\ \hline
123452244         & 0,595316                  & 0                           & 0,595316               & 56,1564    & 1034,72                   & 92                          & 1221,08                & 386,916    \\ \hline
9398429           & 0,595316                  & 0                           & 0,595316               & 57,3738    & 957,161                   & 135                         & 1230,62                & 375,553    \\ \hline
12321             & 0,595316                  & 0                           & 0,595316               & 57,3672    & 1031,52                   & 91                          & 1215,85                & 234,667    \\ \hline
213566            & 0,595316                  & 0                           & 0,595316               & 60,1728    & 1054,99                   & 136                         & 1330,47                & 220,401    \\ \hline
3939021           & 0,595316                  & 0                           & 0,595316               & 51,028     & 987,182                   & 99                          & 1187,72                & 217,812    \\ \hline
\textbf{Media}    & 0,595316                  & 0                           & 0,595316               & 56,41964   & 1013,1146                 & 110,6                       & 1237,148               & 287,0698   \\ \hline
\end{tabular}
\end{table}


% Please add the following required packages to your document preamble:
% \usepackage{multirow}
\begin{table}[H]
\footnotesize
\begin{tabular}{|c|c|c|c|c|c|c|c|c|}
\hline
\multicolumn{9}{|c|}{\textbf{Resultados obtenidos por el algoritmo AGE-UN en el PAR con 20\% de restricciones}}                                                                                                   \\ \hline
\multirow{2}{*}{} & \multicolumn{4}{c|}{\textbf{Rand}}                                                            & \multicolumn{4}{c|}{\textbf{Newthyroid}}                                                      \\ \cline{2-9} 
                  & \textit{\textbf{Tasa\_C}} & \textit{\textbf{Tasa\_inf}} & \textit{\textbf{Agr,}} & \textbf{T} & \textit{\textbf{Tasa\_C}} & \textit{\textbf{Tasa\_inf}} & \textit{\textbf{Agr,}} & \textbf{T} \\ \hline
123452244         & 0,746683                  & 0                           & 0,746683               & 55,8781    & 279,81                    & 268                         & 691,924                & 130,792    \\ \hline
9398429           & 0,746683                  & 0                           & 0,746683               & 59,8021    & 313,299                   & 0                           & 313,299                & 135,768    \\ \hline
12321             & 0,746683                  & 0                           & 0,746683               & 60,1493    & 313,299                   & 0                           & 313,299                & 77,5305    \\ \hline
213566            & 0,746683                  & 0                           & 0,746683               & 56,7057    & 313,299                   & 0                           & 313,299                & 75,2207    \\ \hline
3939021           & 0,746683                  & 0                           & 0,746683               & 39,838     & 313,299                   & 0                           & 313,299                & 75,6606    \\ \hline
\textbf{Media}    & 0,746683                  & 0                           & 0,746683               & 54,47464   & 306,6012                  & 53,6                        & 389,024                & 98,99436   \\ \hline
\end{tabular}
\end{table}

De nuevo se repite lo que ya vemos en el AGG, luego podemos intuir que esto se debe a la dificultad de los conjuntos de datos y no a los comportamientos de los algoritmos, ya que cambiando la forma de realizar las generaciones y actualizar las poblaciones no está influyendo en como se comportan con los datos. Es cierto que los resultados son distintos, pero las comparaciones entre algoritmos se realizarán más adelante.

\subsubsection{AGE con operador de cruce de  segmento fijo.}

% Please add the following required packages to your document preamble:
% \usepackage{multirow}
\begin{table}[H]
\footnotesize
\begin{tabular}{|c|c|c|c|c|c|c|c|c|}
\hline
\multicolumn{9}{|c|}{\textbf{Resultados obtenidos por el algoritmo AGE-SF en el PAR con 10\% de restricciones}}                                                                                                   \\ \hline
\multirow{2}{*}{} & \multicolumn{4}{c|}{\textbf{Iris}}                                                            & \multicolumn{4}{c|}{\textbf{Ecoli}}                                                           \\ \cline{2-9} 
                  & \textit{\textbf{Tasa\_C}} & \textit{\textbf{Tasa\_inf}} & \textit{\textbf{Agr,}} & \textbf{T} & \textit{\textbf{Tasa\_C}} & \textit{\textbf{Tasa\_inf}} & \textit{\textbf{Agr,}} & \textbf{T} \\ \hline
123452244         & 0,595316                  & 0                           & 0,595316               & 34,0342    & 1023,35                   & 77                          & 1335,29                & 163,435    \\ \hline
9398429           & 0,595316                  & 0                           & 0,595316               & 34,1759    & 1063,18                   & 51                          & 1269,79                & 157,654    \\ \hline
12321             & 0,595316                  & 0                           & 0,595316               & 34,8261    & 1065,84                   & 44                          & 1244,09                & 158,094    \\ \hline
213566            & 0,595316                  & 0                           & 0,595316               & 33,9758    & 1048,43                   & 45                          & 1230,73                & 160,576    \\ \hline
3939021           & 0,595316                  & 0                           & 0,595316               & 33,748     & 1041,73                   & 24                          & 1138,96                & 157,618    \\ \hline
\textbf{Media}    & 0,595316                  & 0                           & 0,595316               & 34,152     & 1048,506                  & 48,2                        & 1243,772               & 159,4754   \\ \hline
\end{tabular}
\end{table}

% Please add the following required packages to your document preamble:
% \usepackage{multirow}
\begin{table}[H]
\footnotesize
\begin{tabular}{|c|c|c|c|c|c|c|c|c|}
\hline
\multicolumn{9}{|c|}{\textbf{Resultados obtenidos por el algoritmo AGE-SF en el PAR con 10\% de restricciones}}                                                                                                   \\ \hline
\multirow{2}{*}{} & \multicolumn{4}{c|}{\textbf{Rand}}                                                            & \multicolumn{4}{c|}{\textbf{Newthyroid}}                                                      \\ \cline{2-9} 
                  & \textit{\textbf{Tasa\_C}} & \textit{\textbf{Tasa\_inf}} & \textit{\textbf{Agr,}} & \textbf{T} & \textit{\textbf{Tasa\_C}} & \textit{\textbf{Tasa\_inf}} & \textit{\textbf{Agr,}} & \textbf{T} \\ \hline
123452244         & 0,746683                  & 0                           & 0,746683               & 33,6354    & 288,636                   & 6                           & 307,093                & 67,0257    \\ \hline
9398429           & 0,746683                  & 0                           & 0,746683               & 34,4358    & 288,636                   & 6                           & 307,093                & 66,4185    \\ \hline
12321             & 0,746683                  & 0                           & 0,746683               & 34,125     & 288,636                   & 6                           & 307,093                & 63,0778    \\ \hline
213566            & 0,746683                  & 0                           & 0,746683               & 34,2755    & 288,636                   & 6                           & 307,093                & 64,5503    \\ \hline
3939021           & 0,746683                  & 0                           & 0,746683               & 33,7706    & 288,636                   & 6                           & 307,093                & 51,5833    \\ \hline
\textbf{Media}    & 0,746683                  & 0                           & 0,746683               & 34,04846   & 288,636                   & 6                           & 307,093                & 62,53112   \\ \hline
\end{tabular}
\end{table}



% Please add the following required packages to your document preamble:
% \usepackage{multirow}
\begin{table}[H]
\footnotesize
\begin{tabular}{|c|c|c|c|c|c|c|c|c|}
\hline
\multicolumn{9}{|c|}{\textbf{Resultados obtenidos por el algoritmo AGE-SF en el PAR con 20\% de restricciones}}                                                                                                   \\ \hline
\multirow{2}{*}{} & \multicolumn{4}{c|}{\textbf{Iris}}                                                            & \multicolumn{4}{c|}{\textbf{Ecoli}}                                                           \\ \cline{2-9} 
                  & \textit{\textbf{Tasa\_C}} & \textit{\textbf{Tasa\_inf}} & \textit{\textbf{Agr,}} & \textbf{T} & \textit{\textbf{Tasa\_C}} & \textit{\textbf{Tasa\_inf}} & \textit{\textbf{Agr,}} & \textbf{T} \\ \hline
123452244         & 0,595316                  & 0                           & 0,595316               & 59,5943    & 1018,05                   & 119                         & 1259,1                 & 245,266    \\ \hline
9398429           & 0,595316                  & 0                           & 0,595316               & 62,0204    & 1061,16                   & 89                          & 1241,44                & 229,565    \\ \hline
12321             & 0,595316                  & 0                           & 0,595316               & 59,479     & 1023,08                   & 103                         & 1231,71                & 235,73     \\ \hline
213566            & 0,595316                  & 0                           & 0,595316               & 55,5874    & 1057,14                   & 46                          & 1150,32                & 228,969    \\ \hline
3939021           & 0,595316                  & 0                           & 0,595316               & 63,2785    & 1116,43                   & 168                         & 1456,73                & 245,612    \\ \hline
\textbf{Media}    & 0,595316                  & 0                           & 0,595316               & 59,99192   & 1055,172                  & 105                         & 1267,86                & 237,0284   \\ \hline
\end{tabular}
\end{table}


% Please add the following required packages to your document preamble:
% \usepackage{multirow}
\begin{table}[H]
\footnotesize
\begin{tabular}{|c|c|c|c|c|c|c|c|c|}
\hline
\multicolumn{9}{|c|}{\textbf{Resultados obtenidos por el algoritmo AGE-SF en el PAR con 20\% de restricciones}}                                                                                                   \\ \hline
\multirow{2}{*}{} & \multicolumn{4}{c|}{\textbf{Rand}}                                                            & \multicolumn{4}{c|}{\textbf{Newthyroid}}                                                      \\ \cline{2-9} 
                  & \textit{\textbf{Tasa\_C}} & \textit{\textbf{Tasa\_inf}} & \textit{\textbf{Agr,}} & \textbf{T} & \textit{\textbf{Tasa\_C}} & \textit{\textbf{Tasa\_inf}} & \textit{\textbf{Agr,}} & \textbf{T} \\ \hline
123452244         & 0,746683                  & 0                           & 0,746683               & 55,9348    & 313,299                   & 0                           & 313,299                & 86,7861    \\ \hline
9398429           & 0,746683                  & 0                           & 0,746683               & 60,0174    & 313,299                   & 0                           & 313,299                & 90,8191    \\ \hline
12321             & 0,746683                  & 0                           & 0,746683               & 49,8215    & 313,299                   & 0                           & 313,299                & 83,6514    \\ \hline
213566            & 0,746683                  & 0                           & 0,746683               & 49,1412    & 313,299                   & 0                           & 313,299                & 87,4083    \\ \hline
3939021           & 0,746683                  & 0                           & 0,746683               & 63,3441    & 313,299                   & 0                           & 313,299                & 82,9621    \\ \hline
\textbf{Media}    & 0,746683                  & 0                           & 0,746683               & 55,6518    & 313,299                   & 0                           & 313,299                & 86,3254    \\ \hline
\end{tabular}
\end{table}

De nuevo sigue ocurriendo lo mismo, sin embargo vemos un cambio, ahora si converge en todas las ejecuciones en el conjunto de datos Newthyroid. Esto puede ser debido a usar el operador de cruce de segmento fijo, que explota las soluciones en lugar de explorar, por lo que la diversidad es un poco menor (aunque sigue siendo mucho más amplia comparada con la búsqueda local, por ejemplo).

\subsubsection{AM con 10 cromosomas y aplicando BLS a todos los cromosomas.}

% Please add the following required packages to your document preamble:
% \usepackage{multirow}
\begin{table}[H]
\footnotesize
\begin{tabular}{|c|c|c|c|c|c|c|c|c|}
\hline
\multicolumn{9}{|c|}{\textbf{Resultados obtenidos por el algoritmo AM-10-1 en el PAR con 10\% de restricciones}}                                                                                                  \\ \hline
\multirow{2}{*}{} & \multicolumn{4}{c|}{\textbf{Iris}}                                                            & \multicolumn{4}{c|}{\textbf{Ecoli}}                                                           \\ \cline{2-9} 
                  & \textit{\textbf{Tasa\_C}} & \textit{\textbf{Tasa\_inf}} & \textit{\textbf{Agr,}} & \textbf{T} & \textit{\textbf{Tasa\_C}} & \textit{\textbf{Tasa\_inf}} & \textit{\textbf{Agr,}} & \textbf{T} \\ \hline
123452244         & 0,595316                  & 0                           & 0,595316               & 30,9978    & 1038,49                   & 39                          & 1196,48                & 221,619    \\ \hline
9398429           & 0,595316                  & 0                           & 0,595316               & 30,7944    & 1160,57                   & 52                          & 1371,23                & 245,11     \\ \hline
12321             & 0,595316                  & 0                           & 0,595316               & 29,1573    & 1043,36                   & 45                          & 1225,66                & 285,234    \\ \hline
213566            & 0,595316                  & 0                           & 0,595316               & 28,0696    & 1040,08                   & 19                          & 1117,05                & 252,064    \\ \hline
3939021           & 0,595316                  & 0                           & 0,595316               & 28,8427    & 1091,18                   & 46                          & 1277,53                & 270,623    \\ \hline
\textbf{Media}    & 0,595316                  & 0                           & 0,595316               & 29,57236   & 1074,736                  & 40,2                        & 1237,59                & 254,93     \\ \hline
\end{tabular}
\end{table}

% Please add the following required packages to your document preamble:
% \usepackage{multirow}
\begin{table}[H]
\footnotesize
\begin{tabular}{|c|c|c|c|c|c|c|c|c|}
\hline
\multicolumn{9}{|c|}{\textbf{Resultados obtenidos por el algoritmo AM-10-1 en el PAR con 10\% de restricciones}}                                                                                                  \\ \hline
\multirow{2}{*}{} & \multicolumn{4}{c|}{\textbf{Rand}}                                                            & \multicolumn{4}{c|}{\textbf{Newthyroid}}                                                      \\ \cline{2-9} 
                  & \textit{\textbf{Tasa\_C}} & \textit{\textbf{Tasa\_inf}} & \textit{\textbf{Agr,}} & \textbf{T} & \textit{\textbf{Tasa\_C}} & \textit{\textbf{Tasa\_inf}} & \textit{\textbf{Agr,}} & \textbf{T} \\ \hline
123452244         & 0,746683                  & 0                           & 0,746683               & 29,7002    & 288,636                   & 6                           & 307,093                & 64,9115    \\ \hline
9398429           & 0,746683                  & 0                           & 0,746683               & 31,076     & 288,636                   & 6                           & 307,093                & 59,2456    \\ \hline
12321             & 0,746683                  & 0                           & 0,746683               & 30,0468    & 288,636                   & 6                           & 307,093                & 67,1702    \\ \hline
213566            & 0,746683                  & 0                           & 0,746683               & 27,1252    & 288,636                   & 6                           & 307,093                & 66,3171    \\ \hline
3939021           & 0,746683                  & 0                           & 0,746683               & 26,9275    & 288,636                   & 6                           & 307,093                & 66,728     \\ \hline
\textbf{Media}    & 0,746683                  & 0                           & 0,746683               & 28,97514   & 288,636                   & 6                           & 307,093                & 64,87448   \\ \hline
\end{tabular}
\end{table}

% Please add the following required packages to your document preamble:
% \usepackage{multirow}
\begin{table}[H]
\footnotesize
\begin{tabular}{|c|c|c|c|c|c|c|c|c|}
\hline
\multicolumn{9}{|c|}{\textbf{Resultados obtenidos por el algoritmo AM-10-1 en el PAR con 20\% de restricciones}}                                                                                                  \\ \hline
\multirow{2}{*}{} & \multicolumn{4}{c|}{\textbf{Iris}}                                                            & \multicolumn{4}{c|}{\textbf{Ecoli}}                                                           \\ \cline{2-9} 
                  & \textit{\textbf{Tasa\_C}} & \textit{\textbf{Tasa\_inf}} & \textit{\textbf{Agr,}} & \textbf{T} & \textit{\textbf{Tasa\_C}} & \textit{\textbf{Tasa\_inf}} & \textit{\textbf{Agr,}} & \textbf{T} \\ \hline
123452244         & 0,595316                  & 0                           & 0,595316               & 56,7613    & 969,865                   & 75                          & 1121,78                & 354,644    \\ \hline
9398429           & 0,595316                  & 0                           & 0,595316               & 53,8831    & 1032,09                   & 74                          & 1181,99                & 391,738    \\ \hline
12321             & 0,595316                  & 0                           & 0,595316               & 54,9345    & 981,339                   & 77                          & 1137,31                & 390,72     \\ \hline
213566            & 0,595316                  & 0                           & 0,595316               & 56,8105    & 1001,62                   & 51                          & 1104,92                & 447,256    \\ \hline
3939021           & 0,595316                  & 0                           & 0,595316               & 56,9273    & 1050,8                    & 58                          & 1168,28                & 419,404    \\ \hline
\textbf{Media}    & 0,595316                  & 0                           & 0,595316               & 55,86334   & 1007,1428                 & 67                          & 1142,856               & 400,7524   \\ \hline
\end{tabular}
\end{table}

% Please add the following required packages to your document preamble:
% \usepackage{multirow}
\begin{table}[H]
\footnotesize
\begin{tabular}{|c|c|c|c|c|c|c|c|c|}
\hline
\multicolumn{9}{|c|}{\textbf{Resultados obtenidos por el algoritmo AM-10-1 en el PAR con 20\% de restricciones}}                                                                                                  \\ \hline
\multirow{2}{*}{} & \multicolumn{4}{c|}{\textbf{Rand}}                                                            & \multicolumn{4}{c|}{\textbf{Newthyroid}}                                                      \\ \cline{2-9} 
                  & \textit{\textbf{Tasa\_C}} & \textit{\textbf{Tasa\_inf}} & \textit{\textbf{Agr,}} & \textbf{T} & \textit{\textbf{Tasa\_C}} & \textit{\textbf{Tasa\_inf}} & \textit{\textbf{Agr,}} & \textbf{T} \\ \hline
123452244         & 0,746683                  & 0                           & 0,746683               & 54,0506    & 313,299                   & 0                           & 313,299                & 95,7825    \\ \hline
9398429           & 0,746683                  & 0                           & 0,746683               & 58,7367    & 313,299                   & 0                           & 313,299                & 112,613    \\ \hline
12321             & 0,746683                  & 0                           & 0,746683               & 57,2196    & 313,299                   & 0                           & 313,299                & 125,194    \\ \hline
213566            & 0,746683                  & 0                           & 0,746683               & 56,6627    & 313,299                   & 0                           & 313,299                & 118,048    \\ \hline
3939021           & 0,746683                  & 0                           & 0,746683               & 57,8803    & 313,299                   & 0                           & 313,299                & 123,738    \\ \hline
\textbf{Media}    & 0,746683                  & 0                           & 0,746683               & 56,90998   & 313,299                   & 0                           & 313,299                & 115,0751   \\ \hline
\end{tabular}
\end{table}


En este caso vemos como de nuevo converge a los mismo mínimos en el nuevo conjunto de datos, ya que aunque hemos intentado no favorecer la explotación con el operador de cruce, si que le estamos aplicando una búsqueda local a todos los elementos de la población, sumado a que esta población es mucho más pequeña que con los otros algoritmos (10 cromosomas en lugar de 50).

\subsubsection{AM con 10 cromosomas y aplicando BLS al 10\% de los cromosomas.}

% Please add the following required packages to your document preamble:
% \usepackage{multirow}
\begin{table}[H]
\footnotesize
\begin{tabular}{|c|c|c|c|c|c|c|c|c|}
\hline
\multicolumn{9}{|c|}{\textbf{Resultados obtenidos por el algoritmo AM-10-0.1 en el PAR con 10\% de restricciones}}                                                                                                \\ \hline
\multirow{2}{*}{} & \multicolumn{4}{c|}{\textbf{Iris}}                                                            & \multicolumn{4}{c|}{\textbf{Ecoli}}                                                           \\ \cline{2-9} 
                  & \textit{\textbf{Tasa\_C}} & \textit{\textbf{Tasa\_inf}} & \textit{\textbf{Agr,}} & \textbf{T} & \textit{\textbf{Tasa\_C}} & \textit{\textbf{Tasa\_inf}} & \textit{\textbf{Agr,}} & \textbf{T} \\ \hline
123452244         & 0,595316                  & 0                           & 0,595316               & 32,5543    & 1037,29                   & 22                          & 1126,41                & 170,305    \\ \hline
9398429           & 0,595316                  & 0                           & 0,595316               & 29,0987    & 1007,24                   & 24                          & 1104,47                & 192,828    \\ \hline
12321             & 0,595316                  & 0                           & 0,595316               & 32,3511    & 1007,32                   & 38                          & 1161,27                & 205,26     \\ \hline
213566            & 0,595316                  & 0                           & 0,595316               & 30,1604    & 1010,8                    & 34                          & 1148,54                & 176,419    \\ \hline
3939021           & 0,595316                  & 0                           & 0,595316               & 31,5382    & 1003,92                   & 27                          & 1113,3                 & 167,053    \\ \hline
\textbf{Media}    & 0,595316                  & 0                           & 0,595316               & 31,14054   & 1013,314                  & 29                          & 1130,798               & 182,373    \\ \hline
\end{tabular}
\end{table}


% Please add the following required packages to your document preamble:
% \usepackage{multirow}
\begin{table}[H]
\footnotesize
\begin{tabular}{|c|c|c|c|c|c|c|c|c|}
\hline
\multicolumn{9}{|c|}{\textbf{Resultados obtenidos por el algoritmo AM-10-0.1 en el PAR con 10\% de restricciones}}                                                                                                \\ \hline
\multirow{2}{*}{} & \multicolumn{4}{c|}{\textbf{Rand}}                                                            & \multicolumn{4}{c|}{\textbf{Newthyroid}}                                                      \\ \cline{2-9} 
                  & \textit{\textbf{Tasa\_C}} & \textit{\textbf{Tasa\_inf}} & \textit{\textbf{Agr,}} & \textbf{T} & \textit{\textbf{Tasa\_C}} & \textit{\textbf{Tasa\_inf}} & \textit{\textbf{Agr,}} & \textbf{T} \\ \hline
123452244         & 0,746683                  & 0                           & 0,746683               & 31,8139    & 288,636                   & 6                           & 307,093                & 67,0763    \\ \hline
9398429           & 0,746683                  & 0                           & 0,746683               & 30,265     & 288,636                   & 6                           & 307,093                & 62,3063    \\ \hline
12321             & 0,746683                  & 0                           & 0,746683               & 32,7576    & 288,636                   & 6                           & 307,093                & 71,302     \\ \hline
213566            & 0,746683                  & 0                           & 0,746683               & 29,9967    & 288,636                   & 6                           & 307,093                & 69,1749    \\ \hline
3939021           & 0,746683                  & 0                           & 0,746683               & 30,344     & 288,636                   & 6                           & 307,093                & 69,6438    \\ \hline
\textbf{Media}    & 0,746683                  & 0                           & 0,746683               & 31,03544   & 288,636                   & 6                           & 307,093                & 67,90066   \\ \hline
\end{tabular}
\end{table}

% Please add the following required packages to your document preamble:
% \usepackage{multirow}
\begin{table}[H]
\footnotesize
\begin{tabular}{|c|c|c|c|c|c|c|c|c|}
\hline
\multicolumn{9}{|c|}{\textbf{Resultados obtenidos por el algoritmo AM-10-0.1 en el PAR con 20\% de restricciones}}                                                                                                \\ \hline
\multirow{2}{*}{} & \multicolumn{4}{c|}{\textbf{Iris}}                                                            & \multicolumn{4}{c|}{\textbf{Ecoli}}                                                           \\ \cline{2-9} 
                  & \textit{\textbf{Tasa\_C}} & \textit{\textbf{Tasa\_inf}} & \textit{\textbf{Agr,}} & \textbf{T} & \textit{\textbf{Tasa\_C}} & \textit{\textbf{Tasa\_inf}} & \textit{\textbf{Agr,}} & \textbf{T} \\ \hline
123452244         & 0,595316                  & 0                           & 0,595316               & 53,1792    & 1026,86                   & 88                          & 1205,11                & 269,138    \\ \hline
9398429           & 0,595316                  & 0                           & 0,595316               & 51,4935    & 1047,74                   & 93                          & 1236,12                & 311,127    \\ \hline
12321             & 0,595316                  & 0                           & 0,595316               & 59,5593    & 1026,32                   & 73                          & 1174,19                & 276,328    \\ \hline
213566            & 0,595316                  & 0                           & 0,595316               & 59,3153    & 1030,99                   & 92                          & 1217,35                & 258,225    \\ \hline
3939021           & 0,595316                  & 0                           & 0,595316               & 59,8578    & 1018,28                   & 71                          & 1162,1                 & 270,382    \\ \hline
\textbf{Media}    & 0,595316                  & 0                           & 0,595316               & 56,68102   & 1030,038                  & 83,4                        & 1198,974               & 277,04     \\ \hline
\end{tabular}
\end{table}



% Please add the following required packages to your document preamble:
% \usepackage{multirow}
\begin{table}[H]
\footnotesize
\begin{tabular}{|c|c|c|c|c|c|c|c|c|}
\hline
\multicolumn{9}{|c|}{\textbf{Resultados obtenidos por el algoritmo AM-10-0.1 en el PAR con 20\% de restricciones}}                                                                                                \\ \hline
\multirow{2}{*}{} & \multicolumn{4}{c|}{\textbf{Rand}}                                                            & \multicolumn{4}{c|}{\textbf{Newthyroid}}                                                      \\ \cline{2-9} 
                  & \textit{\textbf{Tasa\_C}} & \textit{\textbf{Tasa\_inf}} & \textit{\textbf{Agr,}} & \textbf{T} & \textit{\textbf{Tasa\_C}} & \textit{\textbf{Tasa\_inf}} & \textit{\textbf{Agr,}} & \textbf{T} \\ \hline
123452244         & 0,746683                      & 0                           & 0,746683                   & 62,2841       & 313,299                   & 0                           & 313,299                & 102,019    \\ \hline
9398429           & 0,746683                      & 0                           & 0,746683                  & 63,6148       & 313,299                   & 0                           & 313,299                & 112,848    \\ \hline
12321             & 0,746683                      & 0                           & 0,746683                   & 59,912       & 313,299                   & 0                           & 313,299                & 116,361    \\ \hline
213566            & 0,746683                      & 0                           & 0,746683                   & 59,0123          & 313,299                   & 0                           & 313,299                & 114,503    \\ \hline
3939021           & 0,746683                      & 0                           & 0,746683                   & 61,3456          & 313,299                   & 0                           & 313,299                & 119,927    \\ \hline
\textbf{Media}    & 0,746683                      & 0                           & 0,746683                   & 60,63376           & 313,299                   & 0                           & 313,299                & 113,1316   \\ \hline
\end{tabular}
\end{table}


\subsubsection{AM con 10 cromosomas y aplicando BLS al 10\% de los mejores cromosomas.}

% Please add the following required packages to your document preamble:
% \usepackage{multirow}
\begin{table}[H]
\footnotesize
\begin{tabular}{|c|c|c|c|c|c|c|c|c|}
\hline
\multicolumn{9}{|c|}{\textbf{Resultados obtenidos por el algoritmo AM-10-0.1mej en el PAR con 10\% de restricciones}}                                                                                             \\ \hline
\multirow{2}{*}{} & \multicolumn{4}{c|}{\textbf{Iris}}                                                            & \multicolumn{4}{c|}{\textbf{Ecoli}}                                                           \\ \cline{2-9} 
                  & \textit{\textbf{Tasa\_C}} & \textit{\textbf{Tasa\_inf}} & \textit{\textbf{Agr,}} & \textbf{T} & \textit{\textbf{Tasa\_C}} & \textit{\textbf{Tasa\_inf}} & \textit{\textbf{Agr,}} & \textbf{T} \\ \hline
123452244         & 0,595316                  & 0                           & 0,595316               & 31,7784    & 1044,11                   & 26                          & 1149,44                & 164,814    \\ \hline
9398429           & 0,595316                  & 0                           & 0,595316               & 30,7307    & 1008,61                   & 52                          & 1219,27                & 185,589    \\ \hline
12321             & 0,595316                  & 0                           & 0,595316               & 30,7091    & 1017,23                   & 44                          & 1195,48                & 161,199    \\ \hline
213566            & 0,595316                  & 0                           & 0,595316               & 31,4057    & 1031,67                   & 33                          & 1165,36                & 148,7      \\ \hline
3939021           & 0,595316                  & 0                           & 0,595316               & 32,652     & 1009,76                   & 22                          & 1098,88                & 148,932    \\ \hline
\textbf{Media}    & 0,595316                  & 0                           & 0,595316               & 31,45518   & 1022,276                  & 35,4                        & 1165,686               & 161,8468   \\ \hline
\end{tabular}
\end{table}

% Please add the following required packages to your document preamble:
% \usepackage{multirow}
\begin{table}[H]
\footnotesize
\begin{tabular}{|c|c|c|c|c|c|c|c|c|}
\hline
\multicolumn{9}{|c|}{\textbf{Resultados obtenidos por el algoritmo AM-10-0.1mej en el PAR con 10\% de restricciones}}                                                                                             \\ \hline
\multirow{2}{*}{} & \multicolumn{4}{c|}{\textbf{Rand}}                                                            & \multicolumn{4}{c|}{\textbf{Newthyroid}}                                                      \\ \cline{2-9} 
                  & \textit{\textbf{Tasa\_C}} & \textit{\textbf{Tasa\_inf}} & \textit{\textbf{Agr,}} & \textbf{T} & \textit{\textbf{Tasa\_C}} & \textit{\textbf{Tasa\_inf}} & \textit{\textbf{Agr,}} & \textbf{T} \\ \hline
123452244         & 0,746683                  & 0                           & 0,746683               & 31,9085    & 288,636                   & 6                           & 307,093                & 66,5646    \\ \hline
9398429           & 0,746683                  & 0                           & 0,746683               & 31,0092    & 288,636                   & 6                           & 307,093                & 64,5443    \\ \hline
12321             & 0,746683                  & 0                           & 0,746683               & 31,7498    & 288,636                   & 6                           & 307,093                & 70,7502    \\ \hline
213566            & 0,746683                  & 0                           & 0,746683               & 31,6214    & 283,208                   & 95                          & 575,442                & 67,9124    \\ \hline
3939021           & 0,746683                  & 0                           & 0,746683               & 29,3911    & 288,636                   & 6                           & 307,093                & 69,0658    \\ \hline
\textbf{Media}    & 0,746683                  & 0                           & 0,746683               & 31,136     & 287,5504                  & 23,8                        & 360,7628               & 67,76746   \\ \hline
\end{tabular}
\end{table}

% Please add the following required packages to your document preamble:
% \usepackage{multirow}
\begin{table}[H]
\footnotesize
\begin{tabular}{|c|c|c|c|c|c|c|c|c|}
\hline
\multicolumn{9}{|c|}{\textbf{Resultados obtenidos por el algoritmo AM-10-0.1mej en el PAR con 20\% de restricciones}}                                                                                             \\ \hline
\multirow{2}{*}{} & \multicolumn{4}{c|}{\textbf{Iris}}                                                            & \multicolumn{4}{c|}{\textbf{Ecoli}}                                                           \\ \cline{2-9} 
                  & \textit{\textbf{Tasa\_C}} & \textit{\textbf{Tasa\_inf}} & \textit{\textbf{Agr,}} & \textbf{T} & \textit{\textbf{Tasa\_C}} & \textit{\textbf{Tasa\_inf}} & \textit{\textbf{Agr,}} & \textbf{T} \\ \hline
123452244         & 0,595316                  & 0                           & 0,595316               & 44,7121    & 984,316                   & 96                          & 1178,77                & 261,338    \\ \hline
9398429           & 0,595316                  & 0                           & 0,595316               & 44,4636    & 1011,49                   & 85                          & 1183,66                & 280,684    \\ \hline
12321             & 0,595316                  & 0                           & 0,595316               & 58,2337    & 975,394                   & 103                         & 1184,03                & 242,542    \\ \hline
213566            & 0,595316                  & 0                           & 0,595316               & 57,8959    & 1028,42                   & 111                         & 1253,26                & 238,36     \\ \hline
3939021           & 0,595316                  & 0                           & 0,595316               & 59,9682    & 984,199                   & 74                          & 1134,09                & 241,127    \\ \hline
\textbf{Media}    & 0,595316                  & 0                           & 0,595316               & 53,0547    & 996,7638                  & 93,8                        & 1186,762               & 252,8102   \\ \hline
\end{tabular}
\end{table}

% Please add the following required packages to your document preamble:
% \usepackage{multirow}
\begin{table}[H]
\footnotesize
\begin{tabular}{|c|c|c|c|c|c|c|c|c|}
\hline
\multicolumn{9}{|c|}{\textbf{Resultados obtenidos por el algoritmo AM-10-0.1mej en el PAR con 20\% de restricciones}}                                                                                             \\ \hline
\multirow{2}{*}{} & \multicolumn{4}{c|}{\textbf{Rand}}                                                            & \multicolumn{4}{c|}{\textbf{Newthyroid}}                                                      \\ \cline{2-9} 
                  & \textit{\textbf{Tasa\_C}} & \textit{\textbf{Tasa\_inf}} & \textit{\textbf{Agr,}} & \textbf{T} & \textit{\textbf{Tasa\_C}} & \textit{\textbf{Tasa\_inf}} & \textit{\textbf{Agr,}} & \textbf{T} \\ \hline
123452244         & 0,746683                  & 0                           & 0,746683               & 49,2114    & 313,299                   & 0                           & 313,299                & 90,0679    \\ \hline
9398429           & 0,746683                  & 0                           & 0,746683               & 45,8253    & 313,299                   & 0                           & 313,299                & 103,773    \\ \hline
12321             & 0,746683                  & 0                           & 0,746683               & 60,1492    & 313,299                   & 0                           & 313,299                & 83,5632    \\ \hline
213566            & 0,746683                  & 0                           & 0,746683               & 59,8315    & 313,299                   & 0                           & 313,299                & 83,0156    \\ \hline
3939021           & 0,746683                  & 0                           & 0,746683               & 57,8916    & 313,299                   & 0                           & 313,299                & 114,417    \\ \hline
\textbf{Media}    & 0,746683                  & 0                           & 0,746683               & 54,5818    & 313,299                   & 0                           & 313,299                & 94,96734   \\ \hline
\end{tabular}
\end{table}

En este caso ocurre lo mismo que en el apartado anterior, ya que seguimos aplicando la búsqueda local. Más adelante compararemos los distintos algoritmos, discutiendo cual se adapta mejor, y encuentra una mejor solución y si es posible hacerla en poco tiempo.


\newpage

\subsection{Análisis de resultados.	}

\subsubsection{10\% de restricciones}

% Please add the following required packages to your document preamble:
% \usepackage{multirow}
\begin{table}[H]
\footnotesize
\begin{tabular}{|c|c|c|c|c|c|c|c|c|}
\hline
\multicolumn{9}{|c|}{\textbf{Resultados globales para el PAR con un 10\% de restricciones}}                                                                                                                           \\ \hline
\multirow{2}{*}{}     & \multicolumn{4}{c|}{\textbf{Iris}}                                                            & \multicolumn{4}{c|}{\textbf{Ecoli}}                                                           \\ \cline{2-9} 
                      & \textit{\textbf{Tasa\_C}} & \textit{\textbf{Tasa\_inf}} & \textit{\textbf{Agr.}} & \textbf{T} & \textit{\textbf{Tasa\_C}} & \textit{\textbf{Tasa\_inf}} & \textit{\textbf{Agr.}} & \textbf{T} \\ \hline
\textbf{Greedy}       & 0,5976                    & 18,2                        & 1,4155                 & 0,0089     & 1620,404                  & 225,2                       & 2532,7281              & 0,3308     \\ \hline
\textbf{BL}           & 0,5953                    & 0                           & 0,5953                 & 0,033      & 1015,3152                 & 34,4                        & 1154,6755              & 1,0767     \\ \hline
\textbf{AGG-UN}       & 0,595316                  & 0                           & 0,595316               & 33,90718   & 1141,344                  & 85,6                        & 1488,124               & 197,5652   \\ \hline
\textbf{AGG-SF}       & 0,595316                  & 0                           & 0,595316               & 32,2308    & 1161,824                  & 79,2                        & 1482,678               & 218,8612   \\ \hline
\textbf{AGE-UN}       & 0,595316                  & 0                           & 0,595316               & 34,59538   & 1054,44                   & 46,4                        & 1242,416               & 190,2626   \\ \hline
\textbf{AGE-SF}       & 0,595316                  & 0                           & 0,595316               & 34,152     & 1048,506                  & 48,2                        & 1243,772               & 159,4754   \\ \hline
\textbf{AM-10-1}      & 0,595316                  & 0                           & 0,595316               & 29,57236   & 1074,736                  & 40,2                        & 1237,59                & 254,93     \\ \hline
\textbf{AM-10-0.1}    & 0,595316                  & 0                           & 0,595316               & 31,14054   & 1013,314                  & 29                          & 1130,798               & 182,373    \\ \hline
\textbf{AM-10-0.1mej} & 0,595316                  & 0                           & 0,595316               & 31,45518   & 1022,276                  & 35,4                        & 1165,686               & 161,8468   \\ \hline
\end{tabular}
\end{table}


% Please add the following required packages to your document preamble:
% \usepackage{multirow}
\begin{table}[H]
\footnotesize
\begin{tabular}{|c|c|c|c|c|c|c|c|c|}
\hline
\multicolumn{9}{|c|}{\textbf{Resultados globales para el PAR con un 10\% de restricciones}}                                                                                                                           \\ \hline
\multirow{2}{*}{}     & \multicolumn{4}{c|}{\textbf{Rand}}                                                            & \multicolumn{4}{c|}{\textbf{Newthyroid}}                                                      \\ \cline{2-9} 
                      & \textit{\textbf{Tasa\_C}} & \textit{\textbf{Tasa\_inf}} & \textit{\textbf{Agr.}} & \textbf{T} & \textit{\textbf{Tasa\_C}} & \textit{\textbf{Tasa\_inf}} & \textit{\textbf{Agr.}} & \textbf{T} \\ \hline
\textbf{Greedy}       & 0,8522                    & 0                           & 0,8522                 & 0,0053     & 321,448                   & 35                          & 429,113                & 0,013897   \\ \hline
\textbf{BL}           & 0,8522                    & 0                           & 0,8522                 & 0,031      & 288,636                   & 6                           & 307,093                & 0,023454     \\ \hline
\textbf{AGG-UN}       & 0,746683                  & 0                           & 0,746683               & 32,4795    & 291,219                   & 22                          & 358,8944               & 69,89108   \\ \hline
\textbf{AGG-SF}       & 0,746683                  & 0                           & 0,746683               & 32,62382   & 291,219                   & 22                          & 358,8944               & 72,15892   \\ \hline
\textbf{AGE-UN}       & 0,746683                  & 0                           & 0,746683               & 33,77456   & 288,636                   & 6                           & 307,093                & 64,07388   \\ \hline
\textbf{AGE-SF}       & 0,746683                  & 0                           & 0,746683               & 34,04846   & 288,636                   & 6                           & 307,093                & 62,53112   \\ \hline
\textbf{AM-10-1}      & 0,746683                  & 0                           & 0,746683               & 28,97514   & 288,636                   & 6                           & 307,093                & 64,87448   \\ \hline
\textbf{AM-10-0.1}    & 0,746683                  & 0                           & 0,746683               & 31,03544   & 288,636                   & 6                           & 307,093                & 67,90066   \\ \hline
\textbf{AM-10-0.1mej} & 0,746683                  & 0                           & 0,746683               & 31,136     & 287,5504                  & 23,8                        & 360,7628               & 67,76746   \\ \hline
\end{tabular}
\end{table}


Ahora con todos los algoritmos en conjunto, vemos como todos los algoritmos vistos hasta el momento superan con creces los resultados obtenidos por el algoritmo greedy, como era de esperar.

Una de las cosas más significativas en la comparación con respecto a la práctica anterior son los tiempos de ejecución, estos se deben a que no es posible factorizar los algoritmos evolutivos luego el tiempo de ejecución se dispara, llegando a tardar unos 200 segundos con el conjunto más complejo\footnote{Tras realizar la práctica y pasar todos los datos a este documento he conseguido reducir los tiempos de ejecución a prácticamente la mitad con los mismos resultados, por falta de tiempo no he podido volver a realizar las ejecuciones, pero al afectar a todos los algoritmos no he visto la necesidad}.

Para realizar las comparaciones entre los algoritmos usaré el conjunto de datos ecoli, ya que con los otros conjuntos de datos siempre converge a mínimos similares.

Con respecto a los algoritmos, vemos que los algoritmos genéticos estacionarios se comportan mejor que los algoritmos genétivos generacionales, esto es debido a que la exploración de los algoritmos estacionarios es mucho mayor y ninguno de estos algoritmos llega a usar ningún método de explotación, luego al explorar más zona del espacio de búsqueda es capaz de buscar soluciones más distintas, y como es en este caso, encontrar nuevas zonas de mínimos.

El problema de no explotar de los algoritmos genéticos generacionales se hace más patente aún con los resultados de los algoritmos meméticos, ya que usan el mismo esquema que los algoritmos genéticos generacionales, pero cada diez iteraciones aplican una búsqueda local suave, que como vemos arregla este problema e incluso mejora los resultados de los algoritmos genéticos estacionarios, a excepción del algoritmo memético que aplica la búsqueda local a todos los cromosomas, ya que esto hace que converja demasiado rápido las soluciones, y aunque encuentra una buena solución, esta no es tan buena comparada con explotar solo el 10\% de los cromosomas, permitiendo un equilibrio entre exploración/explotación.


\subsubsection{20\% de restricciones}


% Please add the following required packages to your document preamble:
% \usepackage{multirow}
\begin{table}[H]
\footnotesize
\begin{tabular}{|c|c|c|c|c|c|c|c|c|}
\hline
\multicolumn{9}{|c|}{\textbf{Resultados globales para el PAR con un 20\% de restricciones}}                                                                                                                           \\ \hline
\multirow{2}{*}{}     & \multicolumn{4}{c|}{\textbf{Iris}}                                                            & \multicolumn{4}{c|}{\textbf{Ecoli}}                                                           \\ \cline{2-9} 
                      & \textit{\textbf{Tasa\_C}} & \textit{\textbf{Tasa\_inf}} & \textit{\textbf{Agr.}} & \textbf{T} & \textit{\textbf{Tasa\_C}} & \textit{\textbf{Tasa\_inf}} & \textit{\textbf{Agr.}} & \textbf{T} \\ \hline
\textbf{Greedy}       & 0,6488                    & 31                          & 1,3451                 & 0,0075     & 1526,108                  & 220,6                       & 1972,9523              & 0,2535     \\ \hline
\textbf{BL}           & 0,5953                    & 0                           & 0,5953                 & 0,0359     & 1003,4532                 & 93,2                        & 1192,2378              & 1,2888     \\ \hline
\textbf{AGG-UN}       & 0,595316                  & 0                           & 0,595316               & 60,29382   & 1033,278                  & 99,2                        & 1234,216               & 326,942    \\ \hline
\textbf{AGG-SF}       & 0,595316                  & 0                           & 0,595316               & 60,2058    & 1035,464                  & 109,4                       & 1257,064               & 327,9496   \\ \hline
\textbf{AGE-UN}       & 0,595316                  & 0                           & 0,595316               & 56,41964   & 1013,1146                 & 110,6                       & 1237,148               & 287,0698   \\ \hline
\textbf{AGE-SF}       & 0,595316                  & 0                           & 0,595316               & 59,99192   & 1055,172                  & 105                         & 1267,86                & 237,0284   \\ \hline
\textbf{AM-10-1}      & 0,595316                  & 0                           & 0,595316               & 55,86334   & 1007,1428                 & 67                          & 1142,856               & 400,7524   \\ \hline
\textbf{AM-10-0.1}    & 0,595316                  & 0                           & 0,595316               & 56,68102   & 1030,038                  & 83,4                        & 1198,974               & 277,04     \\ \hline
\textbf{AM-10-0.1mej} & 0,595316                  & 0                           & 0,595316               & 53,0547    & 996,7638                  & 93,8                        & 1186,762               & 252,8102   \\ \hline
\end{tabular}
\end{table}


% Please add the following required packages to your document preamble:
% \usepackage{multirow}
\begin{table}[H]
\footnotesize
\begin{tabular}{|c|c|c|c|c|c|c|c|c|}
\hline
\multicolumn{9}{|c|}{\textbf{Resultados globales para el PAR con un 20\% de restricciones}}                                                                                                                           \\ \hline
\multirow{2}{*}{}     & \multicolumn{4}{c|}{\textbf{Rand}}                                                            & \multicolumn{4}{c|}{\textbf{Newthyroid}}                                                      \\ \cline{2-9} 
                      & \textit{\textbf{Tasa\_C}} & \textit{\textbf{Tasa\_inf}} & \textit{\textbf{Agr.}} & \textbf{T} & \textit{\textbf{Tasa\_C}} & \textit{\textbf{Tasa\_inf}} & \textit{\textbf{Agr.}} & \textbf{T} \\ \hline
\textbf{Greedy}       & 0,8522                    & 0                           & 0,8522                 & 0,0054     & 328,86                    & 314                         & 811,711                & 0,081181   \\ \hline
\textbf{BL}           & 0,8522                    & 0                           & 0,8522                 & 0,031      & 313,299                   & 0                           & 313,299                & 1,273968   \\ \hline
\textbf{AGG-UN}       & 0,746683                  & 0                           & 0,746683               & 59,18074   & 313,299                   & 0                           & 313,299                & 105,62468  \\ \hline
\textbf{AGG-SF}       & 0,746683                  & 0                           & 0,746683               & 60,97666   & 313,299                   & 0                           & 313,299                & 122,6      \\ \hline
\textbf{AGE-UN}       & 0,746683                  & 0                           & 0,746683               & 54,47464   & 306,6012                  & 53,6                        & 389,024                & 98,99436   \\ \hline
\textbf{AGE-SF}       & 0,746683                  & 0                           & 0,746683               & 55,6518    & 313,299                   & 0                           & 313,299                & 86,3254    \\ \hline
\textbf{AM-10-1}      & 0,746683                  & 0                           & 0,746683               & 56,90998   & 313,299                   & 0                           & 313,299                & 115,0751   \\ \hline
\textbf{AM-10-0.1}    &  0,746683               & 0                    &  0,746683                    & 60,63376      & 313,299                   & 0                           & 313,299                & 113,1316   \\ \hline
\textbf{AM-10-0.1mej} & 0,746683                  & 0                           & 0,746683               & 54,5818    & 313,299                   & 0                           & 313,299                & 94,96734   \\ \hline
\end{tabular}
\end{table}

En este caso vuelve a ocurrir lo mismo con respecto a los comportamientos de los algoritmos. Destacar que como vemos el tiempo de ejecución se duplica, y esto es debido a que, tras hacer un profiling usando la herramienta gprof, la operación que más tiempo ocupa es comprobar la infactibilidad, y en este caso al tener el doble de restricciones, tarda el doble, luego es normal estos tiempos de ejecución. 

\newpage

\subsection{Otros experimentos realizados.}

\subsubsection{Algoritmos meméticos con mayor población.}

Los algoritmos meméticos ejecutados hasta ahora tenían un tamaño de población de 10 elementos, lo que apenas nos daba diversidad, y usando la búsqueda local suave. En este experimento ejecutaré el algoritmo memético con 50 cromosomas en la población y usando la búsqueda local usada en la práctica 1, con esto intentaré ampliar la exploración al tener una población mayor, y a la vez ampliar la explotación al usar una búsqueda local completa y no suave.

El algoritmos se ha ejecutado con 500.000 evaluaciones ya que la búsqueda local de la práctica 1 consume muchas evaluaciones, pero este detalle veremos más adelante en el estudio del comportamiento de los algoritmos que no influirá, ya que convergerá muy pronto, e incluso con menos evaluaciones consiga los mismos resultados.

Aplicaremos la búsqueda local cada 50 generaciones, ya que el tamaño de la población es de 50, por lo que el código de implementación explicado anteriormente sigue funcionando, la única variación es que en el siguiente enumerado enumerado:

\begin{lstlisting}
enum class tipo_generacion {GENERACIONAL, ESTACIONARIO, MEMETICO_1, MEMETICO_0_1, MEMETICO_0_1_MEJ, MEMETICO_BL_1, MEMETICO_BL_0_1, MEMETICO_BL_0_1_MEJ};
\end{lstlisting}

Y en el condicional en el que entrabamos para hacer la búsqueda local suave en los meméticos originales, tendremos en cuenta esta variación de los algoritmos meméticos.

En este caso comentaré directamente la tabla de resultados globales.

% Please add the following required packages to your document preamble:
% \usepackage{multirow}
\begin{table}[H]
\footnotesize
\begin{tabular}{|c|c|c|c|c|c|c|c|c|}
\hline
\multicolumn{9}{|c|}{\textbf{Resultados obtenidos para AM-50-BL-1 en el PAR con 10\% de restricciones}}                                                                                                           \\ \hline
\multirow{2}{*}{} & \multicolumn{4}{c|}{\textbf{Iris}}                                                            & \multicolumn{4}{c|}{\textbf{Ecoli}}                                                           \\ \cline{2-9} 
                  & \textit{\textbf{Tasa\_C}} & \textit{\textbf{Tasa\_inf}} & \textit{\textbf{Agr,}} & \textbf{T} & \textit{\textbf{Tasa\_C}} & \textit{\textbf{Tasa\_inf}} & \textit{\textbf{Agr,}} & \textbf{T} \\ \hline
123452244         & 0,595316                  & 0                           & 0,595316               & 28,7011    & 1007,3                    & 21                          & 1092,38                & 112,629    \\ \hline
9398429           & 0,595316                  & 0                           & 0,595316               & 29,3197    & 992,281                   & 18                          & 1065,2                 & 117,941    \\ \hline
12321             & 0,595316                  & 0                           & 0,595316               & 28,4768    & 989,872                   & 24                          & 1087,1                 & 110,159    \\ \hline
213566            & 0,595316                  & 0                           & 0,595316               & 28,1064    & 1010,69                   & 15                          & 1071,46                & 110,576    \\ \hline
3939021           & 0,595316                  & 0                           & 0,595316               & 30,4754    & 1047,1                    & 15                          & 1107,87                & 110,498    \\ \hline
\textbf{Media}    & 0,595316                  & 0                           & 0,595316               & 29,01588   & 1009,4486                 & 18,6                        & 1084,802               & 112,3606   \\ \hline
\end{tabular}
\end{table}


% Please add the following required packages to your document preamble:
% \usepackage{multirow}
\begin{table}[H]
\footnotesize
\begin{tabular}{|c|c|c|c|c|c|c|c|c|}
\hline
\multicolumn{9}{|c|}{\textbf{Resultados obtenidos para AM-50-BL-1 en el PAR con 10\% de restricciones}}                                                                                                           \\ \hline
\multirow{2}{*}{} & \multicolumn{4}{c|}{\textbf{Rand}}                                                            & \multicolumn{4}{c|}{\textbf{Newthyroid}}                                                      \\ \cline{2-9} 
                  & \textit{\textbf{Tasa\_C}} & \textit{\textbf{Tasa\_inf}} & \textit{\textbf{Agr,}} & \textbf{T} & \textit{\textbf{Tasa\_C}} & \textit{\textbf{Tasa\_inf}} & \textit{\textbf{Agr,}} & \textbf{T} \\ \hline
123452244         & 0,746683                  & 0                           & 0,746683               & 29,7002    & 288,636                   & 6                           & 307,093                & 43,5909    \\ \hline
9398429           & 0,746683                  & 0                           & 0,746683               & 31,076     & 288,636                   & 6                           & 307,093                & 47,8596    \\ \hline
12321             & 0,746683                  & 0                           & 0,746683               & 30,0468    & 288,636                   & 6                           & 307,093                & 46,3276    \\ \hline
213566            & 0,746683                  & 0                           & 0,746683               & 27,1252    & 288,636                   & 6                           & 307,093                & 45,7709    \\ \hline
3939021           & 0,746683                  & 0                           & 0,746683               & 26,9275    & 288,636                   & 6                           & 307,093                & 48,0304    \\ \hline
\textbf{Media}    & 0,746683                  & 0                           & 0,746683               & 28,97514   & 288,636                   & 6                           & 307,093                & 46,31588   \\ \hline
\end{tabular}
\end{table}

% Please add the following required packages to your document preamble:
% \usepackage{multirow}
\begin{table}[H]
\footnotesize
\begin{tabular}{|c|c|c|c|c|c|c|c|c|}
\hline
\multicolumn{9}{|c|}{\textbf{Resultados obtenidos para AM-50-BL-1 en el PAR con 20\% de restricciones}}                                                                                                           \\ \hline
\multirow{2}{*}{} & \multicolumn{4}{c|}{\textbf{Iris}}                                                            & \multicolumn{4}{c|}{\textbf{Ecoli}}                                                           \\ \cline{2-9} 
                  & \textit{\textbf{Tasa\_C}} & \textit{\textbf{Tasa\_inf}} & \textit{\textbf{Agr,}} & \textbf{T} & \textit{\textbf{Tasa\_C}} & \textit{\textbf{Tasa\_inf}} & \textit{\textbf{Agr,}} & \textbf{T} \\ \hline
123452244         & 0,595316                  & 0                           & 0,595316               & 47,1885    & 984,84                    & 58                          & 1102,32                & 125,717    \\ \hline
9398429           & 0,595316                  & 0                           & 0,595316               & 47,815     & 977,055                   & 62                          & 1102,64                & 138,389    \\ \hline
12321             & 0,595316                  & 0                           & 0,595316               & 48,5272    & 957,828                   & 61                          & 1081,39                & 134,175    \\ \hline
213566            & 0,595316                  & 0                           & 0,595316               & 45,1665    & 956,796                   & 71                          & 1100,61                & 129,589    \\ \hline
3939021           & 0,595316                  & 0                           & 0,595316               & 45,6116    & 998,174                   & 52                          & 1103,5                 & 130,586    \\ \hline
\textbf{Media}    & 0,595316                  & 0                           & 0,595316               & 46,86176   & 974,9386                  & 60,8                        & 1098,092               & 131,6912   \\ \hline
\end{tabular}
\end{table}

% Please add the following required packages to your document preamble:
% \usepackage{multirow}
\begin{table}[H]
\footnotesize
\begin{tabular}{|c|c|c|c|c|c|c|c|c|}
\hline
\multicolumn{9}{|c|}{\textbf{Resultados obtenidos para AM-50-BL-1 en el PAR con 20\% de restricciones}}                                                                                                           \\ \hline
\multirow{2}{*}{} & \multicolumn{4}{c|}{\textbf{Rand}}                                                            & \multicolumn{4}{c|}{\textbf{Newthyroid}}                                                      \\ \cline{2-9} 
                  & \textit{\textbf{Tasa\_C}} & \textit{\textbf{Tasa\_inf}} & \textit{\textbf{Agr,}} & \textbf{T} & \textit{\textbf{Tasa\_C}} & \textit{\textbf{Tasa\_inf}} & \textit{\textbf{Agr,}} & \textbf{T} \\ \hline
123452244         & 0,746683                  & 0                           & 0,746683               & 46,0375    & 313,299                   & 0                           & 313,299                & 74,0727    \\ \hline
9398429           & 0,746683                  & 0                           & 0,746683               & 46,5237    & 313,299                   & 0                           & 313,299                & 74,308     \\ \hline
12321             & 0,746683                  & 0                           & 0,746683               & 47,6549    & 313,299                   & 0                           & 313,299                & 73,3863    \\ \hline
213566            & 0,746683                  & 0                           & 0,746683               & 47,5486    & 313,299                   & 0                           & 313,299                & 77,6655    \\ \hline
3939021           & 0,746683                  & 0                           & 0,746683               & 46,5951    & 313,299                   & 0                           & 313,299                & 73,5373    \\ \hline
\textbf{Media}    & 0,746683                  & 0                           & 0,746683               & 46,87196   & 313,299                   & 0                           & 313,299                & 74,59396   \\ \hline
\end{tabular}
\end{table}


% Please add the following required packages to your document preamble:
% \usepackage{multirow}
\begin{table}[H]
\footnotesize
\begin{tabular}{|c|c|c|c|c|c|c|c|c|}
\hline
\multicolumn{9}{|c|}{\textbf{Resultados obtenidos para AM-50-BL-0.1 en el PAR con 10\% de restricciones}}                                                                                                         \\ \hline
\multirow{2}{*}{} & \multicolumn{4}{c|}{\textbf{Iris}}                                                            & \multicolumn{4}{c|}{\textbf{Ecoli}}                                                           \\ \cline{2-9} 
                  & \textit{\textbf{Tasa\_C}} & \textit{\textbf{Tasa\_inf}} & \textit{\textbf{Agr,}} & \textbf{T} & \textit{\textbf{Tasa\_C}} & \textit{\textbf{Tasa\_inf}} & \textit{\textbf{Agr,}} & \textbf{T} \\ \hline
123452244         & 0,595316                  & 0                           & 0,595316               & 99,9317    & 992,281                   & 18                          & 1065,2                 & 210,165    \\ \hline
9398429           & 0,595316                  & 0                           & 0,595316               & 96,5706    & 992,281                   & 18                          & 1065,2                 & 172,834    \\ \hline
12321             & 0,595316                  & 0                           & 0,595316               & 100,094    & 984,949                   & 14                          & 1041,67                & 181,207    \\ \hline
213566            & 0,595316                  & 0                           & 0,595316               & 93,2661    & 992,281                   & 18                          & 1065,2                 & 183,183    \\ \hline
3939021           & 0,595316                  & 0                           & 0,595316               & 100,573    & 1005,52                   & 16                          & 1070,34                & 188,194    \\ \hline
\textbf{Media}    & 0,595316                  & 0                           & 0,595316               & 98,08708   & 993,4624                  & 16,8                        & 1061,522               & 187,1166   \\ \hline
\end{tabular}
\end{table}


% Please add the following required packages to your document preamble:
% \usepackage{multirow}
\begin{table}[H]
\footnotesize
\begin{tabular}{|c|c|c|c|c|c|c|c|c|}
\hline
\multicolumn{9}{|c|}{\textbf{Resultados obtenidos para AM-50-BL-0.1 en el PAR con 10\% de restricciones}}                                                                                                         \\ \hline
\multirow{2}{*}{} & \multicolumn{4}{c|}{\textbf{Rand}}                                                            & \multicolumn{4}{c|}{\textbf{Newthyroid}}                                                      \\ \cline{2-9} 
                  & \textit{\textbf{Tasa\_C}} & \textit{\textbf{Tasa\_inf}} & \textit{\textbf{Agr,}} & \textbf{T} & \textit{\textbf{Tasa\_C}} & \textit{\textbf{Tasa\_inf}} & \textit{\textbf{Agr,}} & \textbf{T} \\ \hline
123452244         & 0,746683                  & 0                           & 0,746683               & 97,5472    & 288,636                   & 6                           & 307,093                & 191,342    \\ \hline
9398429           & 0,746683                  & 0                           & 0,746683               & 91,2788    & 288,636                   & 6                           & 307,093                & 187,318    \\ \hline
12321             & 0,746683                  & 0                           & 0,746683               & 99,2596    & 288,636                   & 6                           & 307,093                & 180,602    \\ \hline
213566            & 0,746683                  & 0                           & 0,746683               & 96,0638    & 288,636                   & 6                           & 307,093                & 186,737    \\ \hline
3939021           & 0,746683                  & 0                           & 0,746683               & 97,6221    & 288,636                   & 6                           & 307,093                & 190,001    \\ \hline
\textbf{Media}    & 0,746683                  & 0                           & 0,746683               & 96,3543    & 288,636                   & 6                           & 307,093                & 187,2      \\ \hline
\end{tabular}
\end{table}


% Please add the following required packages to your document preamble:
% \usepackage{multirow}
\begin{table}[H]
\footnotesize
\begin{tabular}{|c|c|c|c|c|c|c|c|c|}
\hline
\multicolumn{9}{|c|}{\textbf{Resultados obtenidos para AM-50-BL-0.1 en el PAR con 20\% de restricciones}}                                                                                                         \\ \hline
\multirow{2}{*}{} & \multicolumn{4}{c|}{\textbf{Iris}}                                                            & \multicolumn{4}{c|}{\textbf{Ecoli}}                                                           \\ \cline{2-9} 
                  & \textit{\textbf{Tasa\_C}} & \textit{\textbf{Tasa\_inf}} & \textit{\textbf{Agr,}} & \textbf{T} & \textit{\textbf{Tasa\_C}} & \textit{\textbf{Tasa\_inf}} & \textit{\textbf{Agr,}} & \textbf{T} \\ \hline
123452244         & 0,595316                  & 0                           & 0,595316               & 181,779    & 959,519                   & 36                          & 1032,44                & 274,686    \\ \hline
9398429           & 0,595316                  & 0                           & 0,595316               & 176,275    & 972,857                   & 52                          & 1078,19                & 311,326    \\ \hline
12321             & 0,595316                  & 0                           & 0,595316               & 178,28     & 961,767                   & 59                          & 1081,28                & 217,893    \\ \hline
213566            & 0,595316                  & 0                           & 0,595316               & 178,133    & 971,408                   & 45                          & 1062,56                & 257,598    \\ \hline
3939021           & 0,595316                  & 0                           & 0,595316               & 171,321    & 1002,58                   & 55                          & 1113,99                & 308,28     \\ \hline
\textbf{Media}    & 0,595316                  & 0                           & 0,595316               & 177,1576   & 973,6262                  & 49,4                        & 1073,692               & 273,9566   \\ \hline
\end{tabular}
\end{table}


% Please add the following required packages to your document preamble:
% \usepackage{multirow}
\begin{table}[H]
\footnotesize
\begin{tabular}{|c|c|c|c|c|c|c|c|c|}
\hline
\multicolumn{9}{|c|}{\textbf{Resultados obtenidos para AM-50-BL-0.1 en el PAR con 20\% de restricciones}}                                                                                                         \\ \hline
\multirow{2}{*}{} & \multicolumn{4}{c|}{\textbf{Rand}}                                                            & \multicolumn{4}{c|}{\textbf{Newthyroid}}                                                      \\ \cline{2-9} 
                  & \textit{\textbf{Tasa\_C}} & \textit{\textbf{Tasa\_inf}} & \textit{\textbf{Agr,}} & \textbf{T} & \textit{\textbf{Tasa\_C}} & \textit{\textbf{Tasa\_inf}} & \textit{\textbf{Agr,}} & \textbf{T} \\ \hline
123452244         & 0,746683                  & 0                           & 0,746683               & 181,543    & 313,299                   & 0                           & 313,299                & 329,282    \\ \hline
9398429           & 0,746683                  & 0                           & 0,746683               & 179,642    & 313,299                   & 0                           & 313,299                & 325,334    \\ \hline
12321             & 0,746683                  & 0                           & 0,746683               & 180,957    & 313,299                   & 0                           & 313,299                & 338,248    \\ \hline
213566            & 0,746683                  & 0                           & 0,746683               & 180,869    & 313,299                   & 0                           & 313,299                & 284,89     \\ \hline
3939021           & 0,746683                  & 0                           & 0,746683               & 183,078    & 313,299                   & 0                           & 313,299                & 284,707    \\ \hline
\textbf{Media}    & 0,746683                  & 0                           & 0,746683               & 181,2178   & 313,299                   & 0                           & 313,299                & 312,4922   \\ \hline
\end{tabular}
\end{table}



% Please add the following required packages to your document preamble:
% \usepackage{multirow}
\begin{table}[H]
\footnotesize
\begin{tabular}{|c|c|c|c|c|c|c|c|c|}
\hline
\multicolumn{9}{|c|}{\textbf{Resultados obtenidos para AM-50-BL-0.1mej en el PAR con 10\% de restricciones}}                                                                                                      \\ \hline
\multirow{2}{*}{} & \multicolumn{4}{c|}{\textbf{Iris}}                                                            & \multicolumn{4}{c|}{\textbf{Ecoli}}                                                           \\ \cline{2-9} 
                  & \textit{\textbf{Tasa\_C}} & \textit{\textbf{Tasa\_inf}} & \textit{\textbf{Agr,}} & \textbf{T} & \textit{\textbf{Tasa\_C}} & \textit{\textbf{Tasa\_inf}} & \textit{\textbf{Agr,}} & \textbf{T} \\ \hline
123452244         & 0,595316                  & 0                           & 0,595316               & 141,254    & 1009,52                   & 25                          & 1110,8                 & 235,181    \\ \hline
9398429           & 0,595316                  & 0                           & 0,595316               & 136,405    & 1020,45                   & 24                          & 1117,67                & 218,205    \\ \hline
12321             & 0,595316                  & 0                           & 0,595316               & 143,966    & 965,639                   & 25                          & 1066,92                & 216,509    \\ \hline
213566            & 0,595316                  & 0                           & 0,595316               & 140,543    & 1010,37                   & 35                          & 1152,16                & 317,357    \\ \hline
3939021           & 0,595316                  & 0                           & 0,595316               & 143,413    & 988,895                   & 24                          & 1086,12                & 213,575    \\ \hline
\textbf{Media}    & 0,595316                  & 0                           & 0,595316               & 141,1162   & 998,9748                  & 26,6                        & 1106,734               & 240,1654   \\ \hline
\end{tabular}
\end{table}


% Please add the following required packages to your document preamble:
% \usepackage{multirow}
\begin{table}[H]
\footnotesize
\begin{tabular}{|c|c|c|c|c|c|c|c|c|}
\hline
\multicolumn{9}{|c|}{\textbf{Resultados obtenidos para AM-50-BL-0.1mej en el PAR con 10\% de restricciones}}                                                                                                      \\ \hline
\multirow{2}{*}{} & \multicolumn{4}{c|}{\textbf{Rand}}                                                            & \multicolumn{4}{c|}{\textbf{Newthyroid}}                                                      \\ \cline{2-9} 
                  & \textit{\textbf{Tasa\_C}} & \textit{\textbf{Tasa\_inf}} & \textit{\textbf{Agr,}} & \textbf{T} & \textit{\textbf{Tasa\_C}} & \textit{\textbf{Tasa\_inf}} & \textit{\textbf{Agr,}} & \textbf{T} \\ \hline
123452244         & 0,746683                  & 0                           & 0,746683               & 139,921    & 288,636                   & 6                           & 307,093                & 254,896    \\ \hline
9398429           & 0,746683                  & 0                           & 0,746683               & 140,864    & 288,636                   & 6                           & 307,093                & 242,186    \\ \hline
12321             & 0,746683                  & 0                           & 0,746683               & 140,421    & 288,636                   & 6                           & 307,093                & 259,12     \\ \hline
213566            & 0,746683                  & 0                           & 0,746683               & 137,759    & 288,636                   & 6                           & 307,093                & 270,937    \\ \hline
3939021           & 0,746683                  & 0                           & 0,746683               & 138,614    & 288,636                   & 6                           & 307,093                & 263,125    \\ \hline
\textbf{Media}    & 0,746683                  & 0                           & 0,746683               & 139,5158   & 288,636                   & 6                           & 307,093                & 258,0528   \\ \hline
\end{tabular}
\end{table}

% Please add the following required packages to your document preamble:
% \usepackage{multirow}
\begin{table}[H]
\footnotesize
\begin{tabular}{|c|c|c|c|c|c|c|c|c|}
\hline
\multicolumn{9}{|c|}{\textbf{Resultados obtenidos para AM-50-BL-0.1mej en el PAR con 20\% de restricciones}}                                                                                                      \\ \hline
\multirow{2}{*}{} & \multicolumn{4}{c|}{\textbf{Iris}}                                                            & \multicolumn{4}{c|}{\textbf{Ecoli}}                                                           \\ \cline{2-9} 
                  & \textit{\textbf{Tasa\_C}} & \textit{\textbf{Tasa\_inf}} & \textit{\textbf{Agr,}} & \textbf{T} & \textit{\textbf{Tasa\_C}} & \textit{\textbf{Tasa\_inf}} & \textit{\textbf{Agr,}} & \textbf{T} \\ \hline
123452244         & 0,595316                  & 0                           & 0,595316               & 184,28     & 981,213                   & 68                          & 1118,95                & 358,607    \\ \hline
9398429           & 0,595316                  & 0                           & 0,595316               & 239,914    & 1005,54                   & 58                          & 1123,02                & 323,7      \\ \hline
12321             & 0,595316                  & 0                           & 0,595316               & 240,597    & 1001,78                   & 42                          & 1086,85                & 398,436    \\ \hline
213566            & 0,595316                  & 0                           & 0,595316               & 250,833    & 979,485                   & 69                          & 1119,25                & 352,101    \\ \hline
3939021           & 0,595316                  & 0                           & 0,595316               & 251,081    & 977,384                   & 57                          & 1092,84                & 319,909    \\ \hline
\textbf{Media}    & 0,595316                  & 0                           & 0,595316               & 233,341    & 989,0804                  & 58,8                        & 1108,182               & 350,5506   \\ \hline
\end{tabular}
\end{table}


% Please add the following required packages to your document preamble:
% \usepackage{multirow}
\begin{table}[H]
\footnotesize
\begin{tabular}{|c|c|c|c|c|c|c|c|c|}
\hline
\multicolumn{9}{|c|}{\textbf{Resultados obtenidos para AM-50-BL-0.1mej en el PAR con 20\% de restricciones}}                                                                                                      \\ \hline
\multirow{2}{*}{} & \multicolumn{4}{c|}{\textbf{Rand}}                                                            & \multicolumn{4}{c|}{\textbf{Newthyroid}}                                                      \\ \cline{2-9} 
                  & \textit{\textbf{Tasa\_C}} & \textit{\textbf{Tasa\_inf}} & \textit{\textbf{Agr,}} & \textbf{T} & \textit{\textbf{Tasa\_C}} & \textit{\textbf{Tasa\_inf}} & \textit{\textbf{Agr,}} & \textbf{T} \\ \hline
123452244         & 0,746683                  & 0                           & 0,746683               & 226,28     & 313,299                   & 0                           & 313,299                & 302,757    \\ \hline
9398429           & 0,746683                  & 0                           & 0,746683               & 187,135    & 313,299                   & 0                           & 313,299                & 295,085    \\ \hline
12321             & 0,746683                  & 0                           & 0,746683               & 260,251    & 313,299                   & 0                           & 313,299                & 317,933    \\ \hline
213566            & 0,746683                  & 0                           & 0,746683               & 249,812    & 313,299                   & 0                           & 313,299                & 298,736    \\ \hline
3939021           & 0,746683                  & 0                           & 0,746683               & 252,506    & 313,299                   & 0                           & 313,299                & 286,341    \\ \hline
\textbf{Media}    & 0,746683                  & 0                           & 0,746683               & 235,1968   & 313,299                   & 0                           & 313,299                & 300,1704   \\ \hline
\end{tabular}
\end{table}


\textbf{Resultados globales:}

% Please add the following required packages to your document preamble:
% \usepackage{multirow}
\begin{table}[H]
\footnotesize
\begin{tabular}{|c|c|c|c|c|c|c|c|c|}
\hline
\multicolumn{9}{|c|}{\textbf{Resultados globales del PAR con un 10\% de restricciones}}                                                                                                                                  \\ \hline
\multirow{2}{*}{}        & \multicolumn{4}{c|}{\textbf{Iris}}                                                            & \multicolumn{4}{c|}{\textbf{Ecoli}}                                                           \\ \cline{2-9} 
                         & \textit{\textbf{Tasa\_C}} & \textit{\textbf{Tasa\_inf}} & \textit{\textbf{Agr.}} & \textbf{T} & \textit{\textbf{Tasa\_C}} & \textit{\textbf{Tasa\_inf}} & \textit{\textbf{Agr.}} & \textbf{T} \\ \hline
\textbf{Greedy}          & 0,5976                    & 18,2                        & 1,4155                 & 0,0089     & 1620,404                  & 225,2                       & 2532,7281              & 0,3308     \\ \hline
\textbf{BL}              & 0,5953                    & 0                           & 0,5953                 & 0,033      & 1015,3152                 & 34,4                        & 1154,6755              & 1,0767     \\ \hline
\textbf{AGG-UN}          & 0,595316                  & 0                           & 0,595316               & 33,90718   & 1141,344                  & 85,6                        & 1488,124               & 197,5652   \\ \hline
\textbf{AGG-SF}          & 0,595316                  & 0                           & 0,595316               & 32,2308    & 1161,824                  & 79,2                        & 1482,678               & 218,8612   \\ \hline
\textbf{AGE-UN}          & 0,595316                  & 0                           & 0,595316               & 34,59538   & 1054,44                   & 46,4                        & 1242,416               & 190,2626   \\ \hline
\textbf{AGE-SF}          & 0,595316                  & 0                           & 0,595316               & 34,152     & 1048,506                  & 48,2                        & 1243,772               & 159,4754   \\ \hline
\textbf{AM-10-1}         & 0,595316                  & 0                           & 0,595316               & 29,57236   & 1074,736                  & 40,2                        & 1237,59                & 254,93     \\ \hline
\textbf{AM-10-0.1}       & 0,595316                  & 0                           & 0,595316               & 31,14054   & 1013,314                  & 29                          & 1130,798               & 182,373    \\ \hline
\textbf{AM-10-0.1mej}    & 0,595316                  & 0                           & 0,595316               & 31,45518   & 1022,276                  & 35,4                        & 1165,686               & 161,8468   \\ \hline
\textbf{AM-BL-50-1}      & 0,595316                  & 0                           & 0,595316               & 29,01588   & 1009,4486                 & 18,6                        & 1084,802               & 112,3606   \\ \hline
\textbf{AM-BL-50-0.1}    & 0,595316                  & 0                           & 0,595316               & 98,08708   & 993,4624                  & 16,8                        & 1061,522               & 187,1166   \\ \hline
\textbf{AM-BL-50-0.1mej} & 0,595316                  & 0                           & 0,595316               & 141,1162   & 998,9748                  & 26,6                        & 1106,734               & 240,1654   \\ \hline
\end{tabular}
\end{table}



% Please add the following required packages to your document preamble:
% \usepackage{multirow}
\begin{table}[H]
\footnotesize
\begin{tabular}{|c|c|c|c|c|c|c|c|c|}
\hline
\multicolumn{9}{|c|}{\textbf{Resultados globales del PAR con un 10\% de restricciones}}                                                                                                                                  \\ \hline
\multirow{2}{*}{}        & \multicolumn{4}{c|}{\textbf{Rand}}                                                            & \multicolumn{4}{c|}{\textbf{Newthyroid}}                                                      \\ \cline{2-9} 
                         & \textit{\textbf{Tasa\_C}} & \textit{\textbf{Tasa\_inf}} & \textit{\textbf{Agr.}} & \textbf{T} & \textit{\textbf{Tasa\_C}} & \textit{\textbf{Tasa\_inf}} & \textit{\textbf{Agr.}} & \textbf{T} \\ \hline
\textbf{Greedy}          & 0,8522                    & 0                           & 0,8522                 & 0,0053     & 321,448                   & 35                          & 429,113                & 0,013897   \\ \hline
\textbf{BL}              & 0,8522                    & 0                           & 0,8522                 & 0,031      & 288,636                   & 6                           & 307,093                & 0,023454   \\ \hline
\textbf{AGG-UN}          & 0,746683                  & 0                           & 0,746683               & 32,4795    & 291,219                   & 22                          & 358,8944               & 69,89108   \\ \hline
\textbf{AGG-SF}          & 0,746683                  & 0                           & 0,746683               & 32,62382   & 291,219                   & 22                          & 358,8944               & 72,15892   \\ \hline
\textbf{AGE-UN}          & 0,746683                  & 0                           & 0,746683               & 33,77456   & 288,636                   & 6                           & 307,093                & 64,07388   \\ \hline
\textbf{AGE-SF}          & 0,746683                  & 0                           & 0,746683               & 34,04846   & 288,636                   & 6                           & 307,093                & 62,53112   \\ \hline
\textbf{AM-10-1}         & 0,746683                  & 0                           & 0,746683               & 28,97514   & 288,636                   & 6                           & 307,093                & 64,87448   \\ \hline
\textbf{AM-10-0.1}       & 0,746683                  & 0                           & 0,746683               & 31,03544   & 288,636                   & 6                           & 307,093                & 67,90066   \\ \hline
\textbf{AM-10-0.1mej}    & 0,746683                  & 0                           & 0,746683               & 31,136     & 287,5504                  & 23,8                        & 360,7628               & 67,76746   \\ \hline
\textbf{AM-BL-50-1}      & 0,746683                  & 0                           & 0,746683               & 28,97514   & 288,636                   & 6                           & 307,093                & 46,31588   \\ \hline
\textbf{AM-BL-50-0.1}    & 0,746683                  & 0                           & 0,746683               & 96,3543    & 288,636                   & 6                           & 307,093                & 187,2      \\ \hline
\textbf{AM-BL-50-0.1mej} & 0,746683                  & 0                           & 0,746683               & 139,5158   & 288,636                   & 6                           & 307,093                & 258,0528   \\ \hline
\end{tabular}
\end{table}


Vemos como con esta variación conseguimos de media los mejores resultados obtenidos hasta el momento. Con este algoritmo realmente hemos conseguido realizar una  gran exploración por el espacio de búsqueda y una explotación profunda de los mínimos alcanzados.

Vemos como con la version de aplicar la búsqueda local al 10\% de los mejores cromosomas no funciona tan bien, y esto es debido a que la búsqueda local normal explota de forma muy profunda, luego si esa búsqueda se aplica a los mejores elementos estamos sobreexplotando, y reducimos la exploración.

También vemos que los tiempos de los algoritmos son mucho más extraños, y esto ocurre porque la búsqueda local, al estar factorizada, se ejecuta mucho más rápido pero tenemos que recordar que también necesita una cantidad elevada de evaluaciones, de ahí que el algoritmos memético que aplica la búsqueda local a todos los cromosomas sea más rápido, sin embargo generará pocas generaciones.


\newpage


% Please add the following required packages to your document preamble:
% \usepackage{multirow}
\begin{table}[H]
\footnotesize
\begin{tabular}{|c|c|c|c|c|c|c|c|c|}
\hline
\multicolumn{9}{|c|}{\textbf{Resultados globales del PAR con un 20\% de restricciones}}                                                                                                                                  \\ \hline
\multirow{2}{*}{}        & \multicolumn{4}{c|}{\textbf{Iris}}                                                            & \multicolumn{4}{c|}{\textbf{Ecoli}}                                                           \\ \cline{2-9} 
                         & \textit{\textbf{Tasa\_C}} & \textit{\textbf{Tasa\_inf}} & \textit{\textbf{Agr.}} & \textbf{T} & \textit{\textbf{Tasa\_C}} & \textit{\textbf{Tasa\_inf}} & \textit{\textbf{Agr.}} & \textbf{T} \\ \hline
\textbf{Greedy}          & 0,6488                    & 31                          & 1,3451                 & 0,0075     & 1526,108                  & 220,6                       & 1972,9523              & 0,2535     \\ \hline
\textbf{BL}              & 0,5953                    & 0                           & 0,5953                 & 0,0359     & 1003,4532                 & 93,2                        & 1192,2378              & 1,2888     \\ \hline
\textbf{AGG-UN}          & 0,595316                  & 0                           & 0,595316               & 60,29382   & 1033,278                  & 99,2                        & 1234,216               & 326,942    \\ \hline
\textbf{AGG-SF}          & 0,595316                  & 0                           & 0,595316               & 60,2058    & 1035,464                  & 109,4                       & 1257,064               & 327,9496   \\ \hline
\textbf{AGE-UN}          & 0,595316                  & 0                           & 0,595316               & 56,41964   & 1013,1146                 & 110,6                       & 1237,148               & 287,0698   \\ \hline
\textbf{AGE-SF}          & 0,595316                  & 0                           & 0,595316               & 59,99192   & 1055,172                  & 105                         & 1267,86                & 237,0284   \\ \hline
\textbf{AM-10-1}         & 0,595316                  & 0                           & 0,595316               & 55,86334   & 1007,1428                 & 67                          & 1142,856               & 400,7524   \\ \hline
\textbf{AM-10-0.1}       & 0,595316                  & 0                           & 0,595316               & 56,68102   & 1030,038                  & 83,4                        & 1198,974               & 277,04     \\ \hline
\textbf{AM-10-0.1mej}    & 0,595316                  & 0                           & 0,595316               & 53,0547    & 996,7638                  & 93,8                        & 1186,762               & 252,8102   \\ \hline
\textbf{AM-BL-50-1}      & 0,595316                  & 0                           & 0,595316               & 46,86176   & 974,9386                  & 60,8                        & 1098,092               & 131,6912   \\ \hline
\textbf{AM-BL-50-0.1}    & 0,595316                  & 0                           & 0,595316               & 177,1576   & 973,6262                  & 49,4                        & 1073,692               & 273,9566   \\ \hline
\textbf{AM-BL-50-0.1mej} & 0,595316                  & 0                           & 0,595316               & 233,341    & 989,0804                  & 58,8                        & 1108,182               & 350,5506   \\ \hline
\end{tabular}
\end{table}


% Please add the following required packages to your document preamble:
% \usepackage{multirow}
\begin{table}[H]
\footnotesize
\begin{tabular}{|c|c|c|c|c|c|c|c|c|}
\hline
\multicolumn{9}{|c|}{\textbf{Resultados globales del PAR con un 20\% de restricciones}}                                                                                                                                  \\ \hline
\multirow{2}{*}{}        & \multicolumn{4}{c|}{\textbf{Rand}}                                                            & \multicolumn{4}{c|}{\textbf{Newthyroid}}                                                      \\ \cline{2-9} 
                         & \textit{\textbf{Tasa\_C}} & \textit{\textbf{Tasa\_inf}} & \textit{\textbf{Agr.}} & \textbf{T} & \textit{\textbf{Tasa\_C}} & \textit{\textbf{Tasa\_inf}} & \textit{\textbf{Agr.}} & \textbf{T} \\ \hline
\textbf{Greedy}          & 0,8522                    & 0                           & 0,8522                 & 0,0054     & 328,86                    & 314                         & 811,711                & 0,081181   \\ \hline
\textbf{BL}              & 0,8522                    & 0                           & 0,8522                 & 0,031      & 313,299                   & 0                           & 313,299                & 0,273968   \\ \hline
\textbf{AGG-UN}          & 0,746683                  & 0                           & 0,746683               & 59,18074   & 313,299                   & 0                           & 313,299                & 105,62468  \\ \hline
\textbf{AGG-SF}          & 0,746683                  & 0                           & 0,746683               & 60,97666   & 313,299                   & 0                           & 313,299                & 122,6      \\ \hline
\textbf{AGE-UN}          & 0,746683                  & 0                           & 0,746683               & 54,47464   & 306,6012                  & 53,6                        & 389,024                & 98,99436   \\ \hline
\textbf{AGE-SF}          & 0,746683                  & 0                           & 0,746683               & 55,6518    & 313,299                   & 0                           & 313,299                & 86,3254    \\ \hline
\textbf{AM-10-1}         & 0,746683                  & 0                           & 0,746683               & 56,90998   & 313,299                   & 0                           & 313,299                & 115,0751   \\ \hline
\textbf{AM-10-0.1}       & 0,85                      & 0                           & 0,85                   & 0,006      & 313,299                   & 0                           & 313,299                & 113,1316   \\ \hline
\textbf{AM-10-0.1mej}    & 0,746683                  & 0                           & 0,746683               & 54,5818    & 313,299                   & 0                           & 313,299                & 94,96734   \\ \hline
\textbf{AM-BL-50-1}      & 0,746683                  & 0                           & 0,746683               & 46,87196   & 313,299                   & 0                           & 313,299                & 74,59396   \\ \hline
\textbf{AM-BL-50-0.1}    & 0,746683                  & 0                           & 0,746683               & 181,2178   & 313,299                   & 0                           & 313,299                & 312,4922   \\ \hline
\textbf{AM-BL-50-0.1mej} & 0,746683                  & 0                           & 0,746683               & 235,1968   & 313,299                   & 0                           & 313,299                & 300,1704   \\ \hline
\end{tabular}
\end{table}

Con el 20\% de restricciones vuelve a ocurrir lo mismo, y es que se vuelve a producir el mismo caso, al aplicar la búsqueda local al 10\% de los mejores estamos sobreexplotando, y no encuentra una solución tan buena.

\newpage

\subsection{Estudio del comportamiento de los algoritmos genéticos usados.}


Como hemos comentado en el manual de uso, es posible observar como avanzan los algoritmos evolutivos tras las distintas generaciones. En esta sección veremos como se comportan todos los algoritmos evolutivos, no a nivel de resultado, si no en como llegan a las soluciones.

Por no extender más este documento, solo compararé las 5 gráficas (una por semilla) de ecoli con un 10\% de restricciones ya que es el único conjunto de datos suficientemente complejo y donde podremos estudiar mejor el comportamiento ya que será más distinguido. El resto se pueden consultar como explica el manual de uso, aunque los algoritmos se comportan igual que con ecoli, solo que convergen mucho más rápido (es necesario ejecutar el programa para obtener los ficheros con los que se generarán las gráficas).

En estas gráficas podemos ver el número de generación en el eje x, y el valor de la función objetivo del mejor valor en dicha generación. Haremos un análisis de las gráficas al final de estas.

\begin{figure}[H]
  \centering
      \includegraphics[scale = 0.50]{ecoli_set_const_10_12321.png}
 		 \caption{Comportamiento de los algoritmos evolutivos en ecoli 10\% y semilla 12321}
  		\label{fig:g-12321}

\end{figure}


\begin{figure}[H]
  \centering
      \includegraphics[scale = 0.50]{ecoli_set_const_10_213566.png}
 		 \caption{Comportamiento de los algoritmos evolutivos en ecoli 10\% y semilla 213566}
  		\label{fig:g-213566}

\end{figure}

\begin{figure}[H]
  \centering
      \includegraphics[scale = 0.50]{ecoli_set_const_10_3939021.png}
 		 \caption{Comportamiento de los algoritmos evolutivos en ecoli 10\% y semilla 3939021}
  		\label{fig:g-213566}

\end{figure}

\begin{figure}[H]
  \centering
      \includegraphics[scale = 0.50]{ecoli_set_const_10_9398429.png}
 		 \caption{Comportamiento de los algoritmos evolutivos en ecoli 10\% y semilla 9398429}
  		\label{fig:g-213566}

\end{figure}


\begin{figure}[H]
  \centering
      \includegraphics[scale = 0.50]{ecoli_set_const_10_123452244.png}
 		 \caption{Comportamiento de los algoritmos evolutivos en ecoli 10\% y semilla 123452244}
  		\label{fig:g-213566}

\end{figure}

En los cinco gráficos vemos claramente distinguidas 4 tipos de lineas según si la caída es más pronunciada o menos:

\begin{itemize}
	\item Algoritmos genéticos estacionarios.
	\item Algoritmos genéticos generacionales.
	\item Algoritmos meméticos con búsqueda local suave.
	\item Algoritmos meméticos con búsqueda local normal.
\end{itemize}

En los demás gráficos no enseñados en este documento ocurre igual, solo que en otros conjuntos convergen de una forma más rápida y no se aprecia tanto la distinción. Como ya hemos comentado en el análisis de resultados, los mejores resultados son obtenidos por los algoritmos meméticos con la búsqueda local normal, seguidos por los meméticos con la búsqueda local suave, los algoritmos genéticos estacionarios y por último los algoritmos genéticos generacionales.

\newpage

\subsubsection{Algoritmos genéticos estacionarios.}

Estos algoritmos son los que más generaciones generan y tardan más generaciones en converger, de hecho, centrándonos en el caso de ecoli con 10\% de restricciones, mientras que el resto de algoritmos finalizan antes de las 5.000 generaciones los algoritmos genéticos estacionarios llegan a sobrepasar con creces las 35.000 generaciones.

Aunque esto pueda parecer un mal comportamiento del algoritmo es todo lo contrario, si entendemos como funciona nuestro algoritmo genético estacionario, el operador de cruce para estos algoritmos solo se seleccionan dos cromosomas y siempre se cruzan, lo que harán que por generación se evalúen únicamente dos hijos, sumando el número de evaluaciones por mutación, pero el porcentaje de mutación es de tan solo 0.001, mientras que en algoritmos genético generacional se escogen 50 cromosomas y se cruzan un 70\% de estos, que aplicando la esperanza matemática se convierten en unas 35 evaluaciones por generación, luego el número de evaluaciones por generación está bastante alejado.

Como sabemos, al no sustituir la población de forma completa y crear tan pocos hijos por generación estos algoritmos tardan mucho más en sustituir una población por otra totalmente distinta, de forma que favorecen mucho la exploración, pero a cambio de converger de una forma mucho más lenta como vemos. Esto puede ser una ventaja o una desventaja, según el problema, en nuestro caso está suponiendo una ventaja, ya que el conjunto de datos es muy complejo y al realizar esta exploración obtiene mejores resultados que el algoritmo genético generacional.


\subsubsection{Algoritmos genéticos generacionales.}

A diferencia de los algoritmos genéticos estacionarios, este algoritmo en cada nueva generación sustituye completamente a la generación anterior. Esto favorece mucho más la explotación comparado con el algoritmo genético estacionario, ya que al cambiar gran parte de la población y existir elitismo las mejores soluciones se reproducirán y permanecerán en la población, convergiendo de una forma mucho más rápida como vemos en todas las gráficas.

Aún así este hecho puede traernos una desventaja, y es que como vemos, aunque existe más exploración que en otros algoritmos como la búsqueda local (donde la exploración es nula), el converger tan rápido en este problema nos causa la desventaja de que obtenemos mínimos locales peores que el algoritmo genético estacionario.

\newpage

\subsubsection{Algoritmos meméticos con búsqueda local suave.}

Como vemos estos algoritmos son de los que más rápido convergen, si de por si un algoritmo memético tiene la base de un algoritmo genético generacional, sumando el uso de la búsqueda local consumirán aún más evaluaciones lo que causará que no sobreviva muchas generaciones.

Sin embargo, como vemos en esas generaciones se comporta de forma excepcional, siendo capaz de dar muy buenas soluciones, ya que es capaz de aprovechar la exploración de un algoritmo genético generacional sumado a la explotación de una búsqueda local. Sin embargo, este hecho nos hace plantearnos una pregunta, ¿ Si pretendemos explorar, porque no basar el algoritmos memético en un algoritmo genético estacionario que tiene mayor rango de exploración ? Esta pregunta es simple de responder, como hemos visto en la implementación, el algoritmo memético aplicará la búsqueda local cada ciertas generaciones, y el algoritmo genético estacionario requiere muchas generaciones para alterar la mayor parte de su población, luego si nos basamos en este lo único que conseguiremos es hacer que todos los cromosomas converjan de forma rápida por el uso de la búsqueda local, pero al converger el algoritmo genético estacionario no podrá explorar, al cruzar dos cromosomas idénticos, en resumen, la búsqueda local ``adelantaría'' al algoritmo genético estacionario y haría que se estanque.

\subsubsection{Algoritmos meméticos con búsqueda local normal.}

Estos algoritmos  son los que más rápido convergen como podemos ver en la gráfica. Este comportamiento es normal ya que aumentamos la explotación al aplicar una búsqueda local normal en lugar de una suave, y aun así vemos como obtenemos los mejores resultados, ya que también combinamos la exploración que nos ofrecen los algoritmos genéticos.

Este experimento tampoco habría tenido sentido si el tamaño de la población fuera de tan solo diez elementos, como nos pedían para el algoritmo original, ya que como podemos observar en las gráficas, la búsqueda local va a aplicar una fuerte explotación.

De hecho, en la gráfica, a pesar de aparecer en la leyenda el algoritmo memético de cincuenta cromosomas al que aplica la búsqueda local a toda la población no podemos apreciar su linea en la gráfica debido a que al aplicar la búsqueda local a las 50 soluciones no dispone de suficientes evaluaciones.

\newpage

\subsubsection{Diferencias entre los operadores de cruce.}

En este apartado comentaremos las diferencias entre los dos operadores de cruce usados, el operador de cruce uniforme y el operador de cruce de segmento fijo.

Como vemos en la mayoría de casos (tanto en las gráficas como en los resultados medios de las tablas), con el operador de cruce de segmento fijo estamos obteniendo mejores resultados, y es que tenemos que recordad que en la implementación de este operador decidimos que el padre del que el hijo obtiene el segmento fijo es el padre con mejor valoración, favoreciendo la explotación, esto lo podemos observar en las caidas de las curvas, que vemos como en general la curva de los algoritmos que usan este operador están por encima de las curvas que usan el operador de cruce uniforme.

Esto en el algoritmo genético estacionario hace que se logre un mayor equilibrio y obtengamos una mejor solución que en el algoritmo genético generacional, ya que como hemos comentado, al converger más rápido no cuenta con tanta explotación.

Con respecto al operador de cruce uniforme, este operador de cruce no da prioridad a ninguna de los cromosomas, luego se podría decir que es imparcial, y simplemente genera nuevos hijos, dejando la tarea de explorar/explotar a otros operadores como el de selección o si existe elitismo.

En nuestro problema el operador de cruce de segmento fijo es claramente superior y esto se debe a la dificultad del caso, en iris, rand o newthyroid no tenia sentido esta comparación ya que en todos esos conjuntos, con ambos operadores se alcanzaba la misma solución, sin embargo en ecoli, donde las soluciones son mucho más diversas, es donde podemos decidir que operador es más útil para este problema. De nuevo, esto no quiere decir que el operador de cruce uniforme sea malo para todos los casos o problemas, simplemente se adapta peor que el operador de cruce de segmento fijo en este problema.

\end{document}
