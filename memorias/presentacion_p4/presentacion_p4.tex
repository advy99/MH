\documentclass{beamer}
\usepackage[spanish]{babel}
\usetheme{metropolis}           % Use metropolis theme


\usepackage[default]{sourcesanspro}

\usepackage[scale=2]{ccicons}


\hypersetup{
    colorlinks=true,
    linkcolor=black,
    filecolor=magenta,
    urlcolor=cyan,
}

\title{MH: Práctica 4\\
			PAR - Estudio de metaheurística propia}


\date{\today}
\author{Antonio David Villegas Yeguas}
\institute{Universidad de Granada\\
\medskip
\textit{advy99@correo.ugr.es}
}


\begin{document}

 \maketitle

\begin{frame}{Índice}
\tableofcontents
\end{frame}
  
  
\section{Inspiración}
\begin{frame}{Inspiración: Filosofía de Nietzsche}
    
\end{frame}
  
  
  
\section{Propuesta}
  
\section{Implementación}
  
\section{Más información}
  
\begin{frame}{Más información: Código y documentación}

	Todo el código esta disponible en: 
	
	\begin{center}
		\url{https://github.com/advy99/MH}	
	\end{center}
	
	Mientras que te puedes descargar cada práctica con su respectiva documentación en:
	
	\begin{center}
		\url{https://github.com/advy99/MH/releases}	
	\end{center}

	
\end{frame}	  
  
\begin{frame}{Más información: Licencias}
  
	Toda la documentación se encuentra sobre la licencia
 	\href{https://creativecommons.org/licenses/by-nc-sa/4.0/deed.es}{Creative Commons
	Reconocimiento NoCommercial-CompartirIgual 4.0}.

	\begin{center}\ccbyncsa\end{center}

	\vspace{1cm}

	Mientras que el código se encuentra bajo la licencia \href{https://www.gnu.org/licenses/old-licenses/gpl-2.0.html}{GNU GPLv2}
  
\end{frame}


\end{document}